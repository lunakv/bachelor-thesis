%%% Hlavní soubor. Zde se definují základní parametry a odkazuje se na ostatní části. %%%

%% Verze pro jednostranný tisk:
% Okraje: levý 40mm, pravý 25mm, horní a dolní 25mm
% (ale pozor, LaTeX si sám přidává 1in)
\documentclass[12pt,a4paper]{report}
\setlength\textwidth{145mm}
\setlength\textheight{247mm}
\setlength\oddsidemargin{15mm}
\setlength\evensidemargin{15mm}
\setlength\topmargin{0mm}
\setlength\headsep{0mm}
\setlength\headheight{0mm}
% \openright zařídí, aby následující text začínal na pravé straně knihy
\let\openright=\clearpage

%% Pokud tiskneme oboustranně:
% \documentclass[12pt,a4paper,twoside,openright]{report}
% \setlength\textwidth{145mm}
% \setlength\textheight{247mm}
% \setlength\oddsidemargin{14.2mm}
% \setlength\evensidemargin{0mm}
% \setlength\topmargin{0mm}
% \setlength\headsep{0mm}
% \setlength\headheight{0mm}
% \let\openright=\cleardoublepage

%% Vytváříme PDF/A-2u
\usepackage[a-2u]{pdfx}

%% Přepneme na českou sazbu 
\usepackage[czech]{babel}
\usepackage[T1]{fontenc}
\usepackage{textcomp}

%% Použité kódování znaků: obvykle latin2, cp1250 nebo utf8:
\usepackage[utf8]{inputenc}

%%% Další užitečné balíčky (jsou součástí běžných distribucí LaTeXu)
\usepackage{amsmath}        % rozšíření pro sazbu matematiky
\usepackage{amsfonts}       % matematické fonty
\usepackage{amsthm}         % sazba vět, definic apod.
\usepackage{bbding}         % balíček s nejrůznějšími symboly
			    % (čtverečky, hvězdičky, tužtičky, nůžtičky, ...)
\usepackage{bm}             % tučné symboly (příkaz \bm)
\usepackage{graphicx}       % vkládání obrázků
\graphicspath{ {./img/} }
\usepackage{fancyvrb}       % vylepšené prostředí pro strojové písmo
\usepackage{indentfirst}    % zavede odsazení 1. odstavce kapitoly
\usepackage[nottoc]{tocbibind} % zajistí přidání seznamu literatury,
                            % obrázků a tabulek do obsahu
\usepackage{icomma}         % inteligetní čárka v matematickém módu
\usepackage{dcolumn}        % lepší zarovnání sloupců v tabulkách
\usepackage{booktabs}       % lepší vodorovné linky v tabulkách
\usepackage{paralist}       % lepší enumerate a itemize
\usepackage{xcolor}         % barevná sazba

%% Změna písem na Libertine/Biolinum
\usepackage{libertinus}
\usepackage[libertine]{newtxmath}	%% Libertine v matematickém módu
\usepackage{inconsolata}

%% Úprava sazby pro lepší optický kerning a méně zalomených řádků
\usepackage{microtype}	

%%% Údaje o práci

% Název práce v jazyce práce (přesně podle zadání)
\def\NazevPrace{STP řešič pro OpenSMT}

% Název práce v angličtině
\def\NazevPraceEN{STP solver for OpenSMT}

% Jméno autora
\def\AutorPrace{Václav Luňák}

% Rok odevzdání
\def\RokOdevzdani{2020}

% Název katedry nebo ústavu, kde byla práce oficiálně zadána
% (dle Organizační struktury MFF UK, případně plný název pracoviště mimo MFF)
\def\Katedra{Katedra distribuovaných a~spolehlivých systémů}
\def\KatedraEN{Department of distributed and dependable systems}

% Jedná se o katedru (department) nebo o ústav (institute)?
\def\TypPracoviste{Katedra}
\def\TypPracovisteEN{Department}

% Vedoucí práce: Jméno a příjmení s~tituly
\def\Vedouci{doc.~RNDr.~Jan Kofroň,~Ph.D.}

% Pracoviště vedoucího (opět dle Organizační struktury MFF)
\def\KatedraVedouciho{Katedra distribuovaných a~spolehlivých systémů}
\def\KatedraVedoucihoEN{Department of distributed and dependable systems}

% Studijní program a obor
\def\StudijniProgram{Informatika}
\def\StudijniObor{Obecná informatika}

% Nepovinné poděkování (vedoucímu práce, konzultantovi, tomu, kdo
% zapůjčil software, literaturu apod.)
\def\Podekovani{%
Poděkování.
}

% Abstrakt (doporučený rozsah cca 80-200 slov; nejedná se o zadání práce)
\def\Abstrakt{%
Abstrakt.
}
\def\AbstraktEN{%
Abstract.
}

% 3 až 5 klíčových slov (doporučeno), každé uzavřeno ve složených závorkách
\def\KlicovaSlova{%
	{SMT} {OpenSMT} {STP} {diferenční logika} {DPLL(T)}
}
\def\KlicovaSlovaEN{%
	{SMT} {OpenSMT} {STP} {difference logic} {DPLL(T)}
}

%% Balíček hyperref, kterým jdou vyrábět klikací odkazy v PDF,
%% ale hlavně ho používáme k uložení metadat do PDF (včetně obsahu).
%% Většinu nastavítek přednastaví balíček pdfx.
\hypersetup{unicode}
\hypersetup{breaklinks=true}

%% Definice různých užitečných maker (viz popis uvnitř souboru)
\include{makra}

%% Titulní strana a různé povinné informační strany
\begin{document}
\include{titulka}

%%% Strana s automaticky generovaným obsahem bakalářské práce

\tableofcontents

%%% Jednotlivé kapitoly práce jsou pro přehlednost uloženy v samostatných souborech
\chapter*{Úvod}
\addcontentsline{toc}{chapter}{Úvod}

V~mnoha informatických oblastech se setkáváme s~potřebou vyjádřit nějakou časovou informaci. Ať už se pohybujeme na poli plánování, formální verifikace nebo paralelního programování, mohou se na naši práci vztahovat časová omezení, na která musíme brát zřetel. Není tedy divu, že s~postupným vývojem těchto disciplín roste potřeba pro efektivní způsoby, kterými bychom tato omezení mohli analyzovat.

Jedním z~nejzákladnějších a~nejběžnějších typů časových omezení jsou takzvaná \emph{rozdílová omezení}. Jak už napovídá jejich název, týkají se časového rozdílu mezi dvěma událostmi. Matematickým formalizmem bychom mohli rozdílové omezení definovat například ve tvaru $X_i - X_j \in [t_1, t_2]$ nebo ekvivalentně jako $X_i - X_j \leq t$. Vzhledem k~relativní jednoduchosti takové podmínky byl problém splnitelnosti konjunkce rozdílových omezení nazván \emph{simple temporal problem} (STP). Přes jejich zdánlivou přímočarost se ale rozdílová omezení ukázala jako dostatečně expresivní pro velkou řadu problémů. Uvažujme například následující situaci, jejíž paralely můžeme nalézt v~oblasti plánování: 

\begin{center}
\begin{minipage}{\textwidth}
Na strojích $s_1 \dots s_n$ potřebujeme provést celkem $m$ úkolů. Každý úkol se skládá z~nějakého množství operací. Každá operace musí proběhnout na konkrétním stroji a~zabere nějaký čas. Operace náležící témuž úkolu navíc musí proběhnout v~pevně daném pořadí. Dokážeme jednotlivé operace rozplánovat mezi všechny stroje tak, aby výpočet skončil před cílovým časem $T$?
\end{minipage}
\end{center}

Příklad převodu takového zadání do tvaru rozdílových omezení ukazuje obrázek \ref{fig:job}. Jistě není těžké představit si reálná využití pro řešení takového problému. Jak si ale můžeme všimnout, už se nejedná o~běžný STP --- v~problému se navíc objevují disjunkce omezení. Naše podmínky obecně odpovídají výrokové formuli (v~konjunktivní normální formě), jejíž atomy jsou rozdílová omezení. Problém splnitelnosti takovýchto formulí se obecně nazývá \emph{satisfiability modulo theories} (SMT) a~je důležitým předmětem zkoumání v~oboru formální verifikace. Přestože je tento problém obecně složitější než STP, ukážeme si, že jeho rozhodnutí jsme stále schopni dosáhnout vhodným využitím postupů na řešení STP.

\begin{figure}[b]
	\centering
	\includegraphics[width=\textwidth]{jobshop.pdf}
	\caption{Příklad převodu plánovacího problému do rozdílových omezení} 
	\label{fig:job}
\end{figure}

V~této práci se budeme věnovat vytvoření STP řešiče pro verifikační framework OpenSMT. Nejprve si přiblížíme některé základní principy z~oblasti SMT řešičů. Následně analyzujeme existující postupy pro řešení STP v~kontextu SMT a~navrhneme podle nich konkrétní algoritmus pro zapojení do OpenSMT. Navržený algoritmus následně implementujeme. Jako součást práce také naši implementaci otestujeme na rozsáhlé sadě problémů a~srovnáme její účinnost s~existujícími SMT řešiči.

\chapter{Vymezení pojmů}
% TODO add chapter intro

\section{Satisfiability Modulo Theories}

Problém splnitelnosti booleovské formule, označovaný též zkratkou SAT, patří k~nejznámějším problémům z~oboru matematické logiky. V~roce 1971 se stal prvním dokazatelně NP-úplným problémem \cite{Cook71} a~v~důsledku toho i~užitečným nástrojem pro teorii složitosti, pomocí něhož lze rozhodovat o~NP-úplnosti dalších problémů.\footnote{Příklady převodů NP-úplných problémů na SAT můžeme nalézt např. v~\cite[kapitola 19]{Mares17}}

V~praktickém použití však SAT naráží na své limitované vyjadřovací schopnosti. Práce s~booleovskými proměnnými, omezenými pouze na dvě různé hodnoty, může komplikovat převod reálného problému do tvaru booleovské formule. Potřeba užití komplexnějších atomů tak vedla ke zobecnění SAT zvanému \emph{Satisfiability modulo theories} (SMT).

Jak název napovídá, SMT rozšiřuje SAT o~jazyk logických teorií (konkrétně teorií predikátové logiky prvního řádu). Máme-li nějakou teorii $T$, pak instancí SMT rozumíme formuli jazyka této teorie. Jiným pohledem můžeme na instanci SMT nahlížet jako na booleovskou formuli, ve které jsme nahradili některé binární proměnné za predikáty obsažené v~$T$. Problému vztaženému k~této konkrétní teorii pak říkáme \emph{SMT s~ohledem na $T$}.

Jako příklady teorií tradičně řešených v~SMT lze uvézt teorii lineární aritmetiky (LA), teorii neinterpretovaných funkcí s~rovností (EUF), či teorii diferenční logiky (DL), kterou se budeme zabývat ve zbytku práce.

\section{Simple Temporal Problem}\label{stp}

Jedním ze základních podproblémů vyskytujícím se v~takřka všech plánovacích problémech je takzvaný \emph{Simple Temporal Problem} (STP). STP poprvé postulovali v~roce 1991 Dechter, Meiri a~Pearl \cite{Dechter91} a~od té doby našel široká využití jak v~informatických oblastech, tak v~oborech od medicíny \cite{Anselma06} po vesmírný let \cite{Fukunaga97}.

Vstupem STP je množina rozdílových omezení, to jest nerovnic tvaru $$x - y \leq c,$$ kde $x$ a~$y$ jsou proměnné a~$c$ je konstanta. V~závislosti na tom, jakou verzi problému řešíme, přitom pracujeme buď s~celočíselnými, nebo s~reálnými hodnotami. Výstupem tohoto problému je pak rozhodnutí, zda existuje takové ohodnocení proměnných, že jsou splněna všechna zadaná omezení. V~rozšíření problému pak můžeme požadovat na výstupu i~nějaké splňující ohodnocení, pokud existuje, případně nalezení pokud možno co nejmenší podmnožiny omezení, která způsobuje nesplnitelnost problému.

Na první pohled se může zdát pevně daný tvar nerovnic příliš omezující, uvědomme si však, že do této formy můžeme převést několik dalších druhů nerovnic. Nejsnáze zahrneme do problému omezení tvaru $x - y = c$; ty stačí jednoduše nahradit nerovnicemi $x - y \leq c$ a~$x - y \geq c$.

Problematické nejsou ani nerovnice typu $\pm x \leq c$. Pro účely takovýchto omezení si zavedeme novou globální proměnnou $zero$, s~jejíž pomocí převedeme předchozí do tvaru $x - zero \leq c$, respektive $zero - x \leq c$. Pokud pak hledáme splňující ohodnocení proměnných, najdeme takové, kde $zero$ je ohodnoceno nulou. Korektnost tohoto postupu zaručuje následující obecně známé tvrzení.

\begin{tvrz}
	Je-li $\sigma$ splňující ohodnocení nějakého STP a~$\delta$ libovolná konstanta, pak ohodnocení $\pi$ definované pro všechny proměnné $x$ jako $\pi(x) = \sigma(x) + \delta$ je také splňující ohodnocení tohoto STP.
\end{tvrz}
\begin{proof}
	Je-li $\sigma$ splňující ohodnocení, pro libovolné rozdílové omezení $x-y \leq c$ v~problému platí $$\pi(x) - \pi(y) = (\sigma(x) + \delta) - (\sigma(y) + \delta) = \sigma(x) - \sigma(y) \leq c,$$ z~čehož plyne, že i~$\pi$ je splňující ohodnocení.
\end{proof}

Můžeme do problému zahrnout taktéž omezení tvaru $x - y < c$. Pro celočíselné proměnné lze tuto nerovnici ekvivalentně zapsat jako $x - y \leq c-1$. V~reálné variantě pak nahradíme nerovnici výrazem $x - y \leq c - \delta$, přičemž nenastavujeme okamžitě konkrétní hodnotu $\delta$, ale udržujeme si ji pouze symbolicky a~určujeme její vhodné dosazení až při výpočtu splňujícího ohodnocení. Tento postup je detailněji popsán v~sekci \ref{int_v_real}. Uvědomme si, že pokud jsme schopni vyjádřit ostré nerovnosti, umíme vyjádřit i~negace neostrých nerovností a~naopak.

V~kontextu SMT pak teorii obsahující výše popsané nerovnice nazveme \emph{teorie diferenční logiky} a~budeme ji značit DL. Celočíselnou variantu této teorie budeme značit jako IDL a~reálnou variantu jako RDL. Nahradíme-li pak v~booleovské formuli některé termy těmito nerovnicemi, ověření splnitelnosti takto vzniklé formule je instancí \emph{SMT problému s~ohledem na DL}.

\section{OpenSMT}

OpenSMT je moderní SMT řešič, jehož vznik a~vývoj zaštiťuje \emph{Formal Verification and Security Lab} univerzity v~Luganu\footnote{\url{http://verify.inf.usi.ch/FVBTR}}. Byl vytvořen jako nástroj sloužící k~výzkumu nových postupů formální verifikace a~k~podpoření používání SMT řešičů společně s~dalšími verifikačními nástroji. Poskytuje proto jednoduchou infrastrukturu pro vytváření nových řešičů teorie, která je použitelná i~pro vývojáře nespecializující se v~oboru formální verifikace. Současně se aktivně vyvíjí jeho druhá verze, OpenSMT2.


\chapter{Analýza}

\section{Fungování SMT řešičů}\label{smt}

Problém splnitelnosti booleovské formule, označovaný též zkratkou SAT, patří k~nejznámějším problémům z~oboru matematické logiky. V~roce 1971 se stal prvním dokazatelně NP-úplným problémem \cite{Cook71} a~v~důsledku toho i~užitečným nástrojem pro teorii složitosti, pomocí něhož lze rozhodovat o~NP-úplnosti dalších problémů.\footnote{Příklady převodů NP-úplných problémů na SAT můžeme nalézt např. v~\cite[kapitola 19]{Mares17}}

V~praktickém použití však SAT naráží na své limitované vyjadřovací schopnosti. Práce s~binárními proměnnými, omezenými pouze na dvě různé hodnoty, může komplikovat převod reálného problému do tvaru booleovské formule. Potřeba užití komplexnějších atomů tak vedla ke zobecnění SAT zvanému \emph{Satisfiability modulo theories} (SMT).

Jak název napovídá, SMT rozšiřuje SAT o~jazyk logických teorií (konkrétně teorií predikátové logiky prvního řádu). máme-li nějakou teorii $t$, pak instancí smt rozumíme formuli jazyka této teorie. jiným pohledem můžeme na instanci smt nahlížet jako na booleovskou formuli, ve které jsme nahradili některé binární proměnné za predikáty obsažené v~$t$. problému vztaženému k~této konkrétní teorii pak říkáme \emph{smt s~ohledem na t}.

jako příklady teorií tradičně řešených v~smt lze uvézt např. teorii lineární aritmetiky (la), teorii neinterpretovaných funkcí s~rovností (euf), či teorii diferenční logiky (dl), kterou se budeme zabývat ve zbytku práce.

vzhledem k~podobnostem mezi sat a~smt není překvapením, že smt řešiče využívají schopností sat řešičů. přechod mezi rozhraním termů teorie a~sat řešičem zpravidla probíhá jedním ze dvou základních způsobů \cite{Nieuwenhuis05}.

první přístup se nazývá \emph{hladový}. hladové smt řešiče operují ve dvou krocích. v~prvním převedou celou vstupní formuli na ekvisplnitelnou booleovskou formuli. druhý krok pak již spočívá jen v~předání této formule existujícímu sat řešiči. pokud bychom tedy pracovali například s~aritmetikou nad osmibitovými čísly, mohli bychom reprezentovat každou proměnnou osmi binárními proměnnými a~aritmetické operace převézt na odpovídající sekvence logických operací.

nespornou výhodou hladových řešičů je možnost použití již existujících metod implementovaných na řešení sat. pro některé teorie se také hladové řešiče ukazují být rychlejší než jejich alternativy. jejich největší problém pak obecně spočívá v~překladu literálů teorie do booleovských formulí. ten musí být zkonstruován samostatně pro každou teorii, a~navíc může v~závislosti na teorii produkovat formule znatelně delší, než byl původní vstup. efektivní převody existují např. pro euf s~omezenou doménou \cite{randal02}, obecně však hladový přístup není příliš rozšířený.

jeho protějškem je takzvaný \emph{líný přístup}. líný přístup se nesnaží měnit strukturu vstupní formule; namísto toho je každý predikát abstrahován pomocí nové binární proměnné. když pak sat řešič rozhodne o~ohodnocení těchto proměnných, oznámí toto rozhodnutí \emph{theory solveru} pro danou teorii. theory solver je schopen určit, zda je dané ohodnocení konzistentní, tzn.~dokáže rozhodnout o~splnitelnosti nějaké konjunkce literálů teorie. sat řešič potom hledá platná ohodnocení, dokud nenajde takové, které theory solver prohlásí za konzistentní.

ve prospěch líných SMT řešičů svědčí fakt, že pro danou teorii často existují dobře známé postupy na ověření konjunkce literálů. Pro LA například můžeme využít metod lineárního programování, theory solver pro DL řeší STP (viz.~\ref{stp}) a~podobně. Mohou však ztrácet efektivitu zejména v~důsledku \emph{slepého prohledávání} \cite{Moura04}, kde hlavní řešič rozhoduje o~hodnotě predikátů, aniž by a~priori věděl o~důsledcích těchto ohodnocení v~rámci teorie, což může vést k~nutnosti vyzkoušet velké množství ohodnocení, než je nalezeno nějaké, které je s~teorií konzistentní.

V~roce 2004 navrhli Gazinger a~kol. nový přístup zvaný \emph{DPLL(T)} \cite{Gazinger04}. DPLL($T$) má koncepčně blíže k~línému vyhodnocování, integruje však těsněji hlavní řešič s~theory solverem. Místo toho, aby využíval theory solveru až po nalezení nějakého ohodnocení, průběžně mu oznamuje dosavadní rozhodnutí a~periodicky se ho ptá na splnitelnost právě dosazené konjunkce. Theory solver pak kromě kontroly splnitelnosti také oznamuje hlavnímu řešiči důsledky této konjunkce. Tím jsme schopni dříve opustit větve rozhodovacího stromu nekonzistentní s~teorií. 

V~jádru tohoto přístupu stojí všeobecný DPLL($X$) engine, využívající DPLL \cite{Davis60} postupu pro SAT řešiče. Tento engine nemusí mít žádné znalosti o~konkrétní teorii. Dosazením theory solveru $Solver_T$ pro danou teorii $T$ za parametr $X$ pak vytvoříme konkrétní instanci DPLL($T$). Hlavní engine komunikuje se $Solver_T$ pomocí následujícího rozhranní \cite{Gazinger04}:

\begin{description}
	\item[Initialize(L: množina literálů).] Inicializuje $Solver_T$ s~$L$ jakožto množinou literálů, které se vyskytují v~problému.
	\item[SetTrue(l: L-literál): množina L-literálů.] Skončí výjimkou, pokud se $l$ ukáže jako nekonzistentní s~dosud zadanými literály teorie. V~opačném případě přidá $l$ do seznamu zadaných literálů a~vrátí množinu L-literálů, které jsou důsledky přidání $l$ do tohoto seznamu.
	\item[IsTrue?(l: L-literál): boolean.] Vrátí \emph{true} právě tehdy, když $l$ je důsledkem seznamu přidaných literálů. \emph{false} tedy vrací, pokud je důsledkem tohoto seznamu $\neg l$, nebo pokud z~něj nevyplývá ani $l$, ani $\neg l$.
	\item[Backtrack(n: přir. číslo).] Odstraní posledních $n$ hodnot ze seznamu zadaných literálů. $n$ nesmí být větší než velikost tohoto seznamu.
	\item[Explain(l: L-literál): množina L-literálů.] Vrátí pokud možno co nejmenší podmnožinu zadaných literálů, z~jejichž konjunkce plyne $l$. Pro $l$ musí platit, že je důsledkem nějaké takové podmnožiny, tedy musí být obsažen v~návratové hodnotě nějakého volání \icode{SetTrue(l')} takového, že $l'$ nebylo zahozeno žádným následným voláním \icode{Backtrack}.
\end{description}

Při použití tohoto rozhranní přitom $Solver_T$ nemusí nic vědět o~implementaci DPLL($X$) enginu. Framework je tedy velice modulární a~snadno rozšiřitelný o~nové teorie. Vyžadujeme pouze, aby byl $Solver_T$ schopný inkrementálně přijímat a~odebírat jednotlivé literály teorie. Tento postup se v~praxi ukazuje jako efektivnější než dostupné alternativy. Většina dnes rozšířených SMT řešičů --- včetně námi používaného OpenSMT2 --- je tedy založena na metodě DPLL($T$).

\section{Rozbor STP}\label{stp}

Jedním ze základních podproblémů vyskytujícím se v~takřka všech plánovacích problémech je takzvaný Simple Temporal Problem (STP). STP poprvé postulovali v~roce 1991 Dechter, Meiri a~Pearl \cite{Dechter91} a~od té doby našel široká využití jak v~informatických oblastech, tak v~oborech od medicíny \cite{Anselma06} po vesmírný let \cite{Fukunaga97}.

Vstupem STP je množina rozdílových omezení, to jest nerovnic tvaru $$x - y \leq c,$$ kde $x$ a~$y$ jsou proměnné a~$c$ je konstanta. V~závislosti na tom, jakou verzi problému řešíme, přitom pracujeme buď s~celočíselnými, nebo s~reálnými hodnotami. Výstupem tohoto problému je pak rozhodnutí, zda existuje ohodnocení proměnných tak, aby byla splněna všechna zadaná omezení. V~rozšíření problému pak můžeme požadovat na výstupu i~nějaké takovéto splnitelné ohodnocení, pokud existuje, případně nalezení pokud možno co nejmenší podmnožiny omezení, která zajišťuje nesplnitelnost problému.

Na první pohled se může zdát pevně daný tvar nerovnic příliš omezující, uvědomme si však, že do této formy můžeme převést několik dalších druhů nerovnic. Nejsnáze zahrneme do problému omezení tvaru $x - y = c$; ty stačí jednoduše nahradit nerovnicemi $x - y \leq c$ a~$x - y \geq c$.

Problematické nejsou ani nerovnice typu $\pm x \leq c$. Pro účely takovýchto omezení si zavedeme novou globální proměnnou $zero$, s~jejíž pomocí převedeme předchozí do tvaru $x - zero \leq c$, respektive $zero - x \leq c$. Pokud pak hledáme splňující ohodnocení proměnných, najdeme takové, kde $zero$ je ohodnoceno nulou. Korektnost tohoto postupu zaručuje následující obecně známé tvrzení.

\begin{tvrz}
	Je-li $\sigma$ splňující ohodnocení nějakého STP a~$\delta$ libovolná konstanta, pak ohodnocení $\pi$ definované pro všechny proměnné $x$ jako $\pi(x) = \sigma(x) + \delta$ je také splňující ohodnocení tohoto STP.
\end{tvrz}
\begin{proof}
	Je-li $\sigma$ splňující ohodnocení, pro libovolné rozdílové omezení $x-y \leq c$ v~problému platí $$\pi(x) - \pi(y) = (\sigma(x) + \delta) - (\sigma(y) + \delta) = \sigma(x) - \sigma(y) \leq c,$$ z~čehož je i~$\pi$ splňující ohodnocení.
\end{proof}

Můžeme do problému zahrnout taktéž omezení tvaru $x - y < c$. Pro celočíselné proměnné lze tuto nerovnici ekvivalentně zapsat jako $x - y \leq c-1$. V~reálné variantě pak nahradíme nerovnici výrazem $x - y \leq c - \delta$, přičemž nenastavujeme okamžitě konkrétní hodnotu $\delta$, ale udržujeme si ji pouze symbolicky a~určujeme její vhodné dosazení až při výpočtu splňujícího ohodnocení. Tento postup je detailněji popsán v~sekci \ref{int_v_real}. Uvědomme si, že pokud jsme schopni vyjádřit ostré nerovnosti, umíme vyjádřit i~negace neostrých nerovností a~naopak.

V~jazyce výrokové logiky pak teorii obsahující výše popsané nerovnice nazveme \emph{teorie diferenční logiky} a~budeme ji značit DL. Celočíselnou variantu této teorie pak budeme značit jako IDL a~reálnou variantu jako RDL. Nahradíme-li pak v~booleovské formuli některé termy těmito nerovnicemi, ověření splnitelnosti takto vzniklé formule je instancí SMT problému s~ohledem na DL.



\section{Převod na grafový problém}\label{graf}

Velkou rozšířenost STP můžeme mimo jiné přisoudit tomu, že jsme schopni ho efektivně řešit. Jelikož se problém skládá výlučně z~lineárních omezení, mohli bychom na první pohled využít metod lineárního programování, jako je například simplexový algoritmus. Tyto metody jsou schopny řešit i~mnohem komplexnější problémy, avšak s~jejich výpočetní silou se pojí znatelně vyšší časová náročnost. Algoritmy specializované na STP se proto už od svého počátku \cite[Kapitola 2]{Dechter91} obracejí jiným směrem, a~to k~formalizmu teorie grafů. Přestože v~průběhu let vzikly různé metody řešení tohoto problému, všechny fungují na základě převedení množiny omezení na takzvaný \emph{omezující graf}.

\begin{definice}[Omezující graf]
	Nechť $\Pi$ je množina rozdílových omezení. Omezujícím grafem této množiny rozumíme hranově ohodnocený orientovaný graf G takový, že vrcholy G tvoří proměnné vyskytující se v~$\Pi$ a~každému omezení $(x-y \leq c) \in \Pi$ odpovídá v~G hrana $\langle x,y\rangle$ s~ohodnocením $c$.
\end{definice}
\begin{pozn}
	Hranu $\langle x,y\rangle$ s~ohodnocením $c$ budeme značit $x \xrightarrow{c} y$. Orientovanou cestu z~$x$ do $y$ se součtem ohodnocení $k$ pak budeme značit $x \xrightarrow{k*} y$.
\end{pozn}

Pro úplnost dodejme, že dvojice proměnných se může vyskytovat v~libovolně mnoha omezeních. Omezující graf je tedy formálně orientovaným multigrafem. Vzhledem k~vzájemné bijekci mezi hranami grafu a~nerovnicemi problému budeme v~průběhu práce volně přecházet mezi oběma reprezentacemi.

Převod do formy grafu je pro řešení problému zásadní. Umožňuje nám totiž formulovat následující klíčové tvrzení.

\begin{tvrz}[Dechter a~kol. \cite{Dechter91}]
	Nechť $\Pi$ je množina rozdílových omezení. Instance STP tvořená touto množinou je splnitelná právě tehdy, když omezující graf $\Pi$ neobsahuje záporné cykly.
\end{tvrz}
\begin{proof}
	Najdeme-li v~omezujícím grafu záporný cyklus obsahující vrchol $x$, sečtením všech nerovnic vyskytujících se v~tomto cyklu dostaneme $x-x \leq c < 0$, z~čehož je jasně problém nesplnitelný. Je-li na druhou stranu problém nesplnitelný, obsahuje $\Pi$ nějakou nerovnici $x - y \leq c$ takovou, že z~$\Pi$ vyplývá $y - x < -c$. Tato implikace znamená, že v~omezujícím grafu existuje cesta $y \xrightarrow{k*} x$ taková, že $k < -c$. Hrana $x \xrightarrow{c} y$ pak společně s~touto cestou tvoří záporný cyklus.
\end{proof}

Hledání splnitelnosti STP jsme tedy schopni převést na hledání záporného cyklu v~grafu. To je problém, který dokážeme efektivně řešit. Využít můžeme např. některý algoritmus na hledání nejkratší cesty, kupříkladu Floydův-Warshallův algoritmus operující v~čase $\Theta(\abs{V}^3)$ nebo Bellmanův-Fordův algoritmus, který má časovou složitost $\Theta(\abs{V}\cdot\abs{E})$.

Tyto algoritmy však trpí pro náš účel zásadním nedostatkem. Jejich použití znamená, že po každém přidání nové hrany do grafu musí znovu proběhnout celé prohledávání. Tento postup není vhodný pro použití v~SMT řešičích, ve kterých je kladen velký důraz na inkrementalitu. V~následující sekci tedy podrobně rozebereme několik postupů pro řešení problémů SMT s~ohledem na DL a~motivujeme výběr námi použitého algoritmu. 

\section{Volba algoritmu}\label{alg}

Jak jsme ukázali v~předchozí sekci, ne všechny algoritmy na rozhodnutí STP jsou dobrou volbou pro použití v~kontextu SMT řešičů. Theory solver pro DPLL($T$) by měl efektivně podporovat dvě zásadní operace --- inkrementální přidávání literálů a~backtracking.

Na rozdíl od základního hladového přístupu, jak byl popsán v~\ref{smt}, v~DPLL($T$) dostává $Solver_T$ informaci o~rozhodnutých ohodnoceních průběžně. Aby byly dříve odhaleny slepé větve v~rozhodovacím stromu, $Solver_T$ je průběžně dotazován na splnitelnost dosud provedených rozhodnutí. Pro zrychlení tohoto procesu je tedy vhodné, aby byl schopen pro výpočet využít výsledků z~předchozích dokončených výpočtů. Algoritmy, které nedokáží takto zakomponovat mezivýpočet podproblému, budou ze své podstaty méně výkonné na postupné sekvenci splnitelných ohodnocení.

Po nalezení nesplnitelného ohodnocení pak neopakuje engine celý výpočet, ale vrací se pouze do nejbližšího stavu, ve kterém bylo ohodnocení ještě splnitelné. Od theory solveru očekáváme, že je schopen efektivně zapomínat přidané literály, vracet se do předchozích stavů a~pokračovat z~nich ve výpočtu. %% FIXME: Better wording?
S~ohledem na tyto požadavky vzniklo několik algoritmů pro řešení SMT s~ohledem na DL. V~této práci se budeme zabývat převážně postupem založeným na vyčerpávající propagaci teorie, který postulovali v~roce 2005 Nieuwenhuis a~Oliveras \cite{Nieuwenhuis05}.

\subsection{DPLL($T$) s~vyčerpávající propagací teorie}

Připomeňme si metodu \icode{SetTrue} uvedenou v~sekci \ref{smt}. Ta slouží k~přidání nového omezení do kontextu řešiče teorie. Řešič může volitelně jako návratovou hodnotu uvést nějakou množinu literálů, které detekoval jako důsledky vzniklé přidáním právě tohoto omezení. Může přitom hlásit např. jen \uv{zjevné} důsledky, popřípadě nemusí vracet vůbec žádné. Pokud jsme je ale schopni nacházet v~rozumém čase, nahlášené důsledky jsou užitečnou informací pro hlavní engine, poněvadž pro něj mohou výrazně zmenšit velikost rozhodovacího stromu.

Varianta DPLL($T$) s~vyčerpávající propagací teorie přidává pro \icode{SetTrue} silnou podmínku; řešič musí vrátit \emph{všechny} literály ze vstupní formule, které jsou důsledky stávajícího ohodnocení. Tento předpoklad značně zjednoduší DPLL($X$) engine a~umožní mu pracovat efektivněji. Stane se z~něj \emph{de~facto} běžný SAT řešič, který se liší pouze minimalistickým rozhranním se $Solver_T$. To má mimo jiné za důsledek, že jsme nyní schopni převézt libovolný SAT řešič na bázi DPLL do DPLL($X$) enginu.

Nevýhoda tohoto přístupu je jasná. Povinnost hledat všechny důsledky teorie může řádově zpomalit operaci \icode{SetTrue} pro řešič teorie. Nicméně se ukázalo, že alespoň v~případě diferenční logiky může tento přístup vést k~rychlé implementaci schopné předčit ostatní alternativy.

\subsection{Návrh řešiče pro diferenční logiku}

$Solver_T$ diferenční logiky využívá principy, se kterými jsme se podrobněji seznámili v~předchozích sekcích. Můžeme například předpokládat, že všechna omezení jsou tvaru $x-y \leq c$ (viz.~\ref{stp}). Využijeme též převodu omezení do tvaru omezujícího grafu, jak bylo uvedeno v~\ref{graf}.

\subsubsection*{Inicializace}
Během inicializace načte řešič vstupní formuli, uloží si všechna rozdílová omezení, která se v~ní vyskytují, a~předá ji DPLL($X$) jako čistě booleovskou formuli. Během tohoto procesu by měl detekovat vztahy mezi literály a~jejich negacemi a~explicitně je poznamenat. Pokud se například na vstupu vyskytnou nerovnice $x-y \leq 1$ a~$y-x \leq -2$, v~oboru celých čísel je jedna negací druhé. Řešič by tak měl abstrahovat tyto výskyty jako $p$ a~$\neg p$ pro booleovskou proměnnou $p$. Ukládá si přitom překladovou tabulku pamatující si pro každou nerovnici abstrahovaný literál, kterému odpovídá. Zároveň s~tím si udržuje pro každou proměnnou seznam všech nerovnic, ve kterých se tato proměnná vyskytuje (účel těchto seznamů je objasněn níže).

\subsubsection*{Ohodnocení literálu}
Jakmile je následně nastavena pravdivostní hodnota některého literálu, převede jej řešič do formy $x-y \leq c$ a~přidá odpovídající hranu do omezujícího grafu. Následně musí objevit všechny důsledky tohoto ohodnocení. Pro jejich nalezení je potřeba zkontrolovat všechny cesty $$x_i \xrightarrow{c_i*} x \xrightarrow{c} y \xrightarrow{c'_j*} y_j$$ a~zjistit, zda nějaké omezení neplyne z~$x_i - y_j \leq (c_i + c + c'_j)$. To budou právě nerovnice tvaru $x_i - y_j \leq c'$, kde $c' \geq c_i + c + c'_j$. 

Abychom prošli všechny tyto cesty, procházející nově přidanou hranou, musíme nejdřív nalézt seznam všech vrcholů $x_i$, ze kterých je dosažitelné $x$, a~seznam všech vrcholů $y_j$, které jsou dosažitelné z~$y$. Omezující graf je tedy reprezentován jako oboustranný seznam sousedů. Potom už můžeme spustit obyčejný algoritmus na hledání nejkratší cesty, kterým získáme všechna vhodná $x_i$ společně s~jejich $c_i$, resp. $y_j$ s~jejich $c'_j$. Autoři pro toto prohledávání doporučují následující postup.

Použijeme běžné prohledávání do hloubky. V~něm si označíme každý vrchol, který jsme navštívili poprvé, společně s~jeho vzdáleností. Navštívíme-li pak již objevený vrchol znovu, pokračujeme v~prohledávání pouze tehdy, když je jeho současná vzdálenost menší než naše uložená vzdálenost. Zároveň si všechny objevené $x_i$ a~$y_j$ ukládáme do dvou odlišných seznamů. Po skončení prohledávání spočteme pro oba seznamy počet všech nerovnic, ve kterých se proměnné v~seznamu vyskytují (tyto nerovnice si pro každou proměnnou pamatujeme z~inicializace).

Vyskytují-li se potom například $x_i$ celkově v~menším počtu proměnných, projdeme pro každé $x_i$ seznam všech nerovnic, ve kterých se vyskytuje, a~zkontrolujeme, zda se nejedná o~důsledek nalezené cesty z~$x_i$ do nějakého $y_j$.

\subsubsection*{Hledání příčin}

Jak víme z~\ref{smt}, řešiče teorie musí implementovat operaci \icode{Explain}, která pro nalezený důsledek vrátí množinu jeho příčin. Tato operace je důležitá pro budování implikačního grafu v~DPLL($X$) a~určení vhodné úrovně backtrackingu při nalezení sporu. Řešič DL popsaný v~\cite{Nieuwenhuis05} tuto operaci provádějí následovně.

Když je do omezujícího grafu přidána $n$-tá hrana, zapamatujeme si k~této hraně její odpovídající $n$. U~nalezených důsledků si pamatujeme $n$ hrany, jejíž přidáním důsledek vznikl. Když pak hledáme vysvětlení hrany $h$ tvaru $x-y \leq c$, spustíme prohledávání z~$x$ do $y$ stejným způsobem jako po přidání hrany. Prohledávání ale pouštíme jen do délky nejvýše $c$ a~jen přes hrany s~číslem vložení nejvýše $n$. Tato omezení zvyšují efektivitu (zmenšujeme prostor k~prohledání) a~zaručují, že nepoužíváme hrany přidané až po důsledku (což by působilo problémy v~implikačním grafu DPLL($X$)). Nalezená cesta tak má délku kratší nebo rovnou $c$ a~sestává se jen z~dříve přidaných hran, z~čehož jasně vidíme, že $h$ je důsledek hran na této cestě.

\subsection{Alternativní řešení}

TODO

\section{Prostředí}

Požadavky na použité prostředí jsou určeny převážně požadavky frameworku OpenSMT, pod nějž tato práce spadá. OpenSMT --- a~tudíž i~tento projekt --- je programován v~jazyce C++, konkrétně ve verzi C++11. Práce byla vyvíjena a~testována na operačním systému s~linuxovým jádrem nad architekturou x64. Jelikož si nejsme vědomi toho, že bychom použili vlastnosti jazyka specifické pro tuto konfiguraci, věříme, že náš kód bude možné bez větších potíží zprovoznit i~na jiných platformách a~operačních systémech.


\chapter{Popis řešení}

V~této kapitole rozebereme obecné vlastnosti našeho řešení. Nejprve detailněji popíšeme použitý algoritmus a~opodstatníme některá koncepční rozhodnutí, která jsme učinili jako důsledek použitého frameworku. Poté nastíníme obecný způsob fungování programu a~zamyslíme se nad rozdíly mezi různými variantami řešeného problému.

\section{DPLL($T$) s~vyčerpávající propagací teorie}\label{dpllt}

Připomeňme si metodu \icode{SetTrue} uvedenou v~sekci \ref{smt}. Ta slouží k~přidání nového omezení do kontextu řešiče teorie. Řešič může volitelně jako návratovou hodnotu uvést nějakou množinu literálů, které detekoval jako důsledky vzniklé přidáním právě tohoto omezení. Může přitom hlásit např. jen \uv{zjevné} důsledky, popřípadě nemusí vracet vůbec žádné. Pokud jsme je ale schopni nacházet v~rozumém čase, nahlášené důsledky jsou užitečnou informací pro hlavní engine, poněvadž pro něj mohou výrazně zmenšit velikost rozhodovacího stromu.

Varianta DPLL($T$) s~vyčerpávající propagací teorie přidává pro \icode{SetTrue} silnou podmínku; řešič musí vrátit \emph{všechny} literály ze vstupní formule, které jsou důsledky stávajícího ohodnocení. Tento předpoklad značně zjednoduší DPLL($X$) engine a~umožní mu pracovat efektivněji. Stane se z~něj \emph{de~facto} běžný SAT řešič, který se liší pouze minimalistickým rozhranním se \Solver. To má mimo jiné za důsledek, že jsme nyní schopni převézt libovolný SAT řešič na bázi DPLL do DPLL($X$) enginu.

Nevýhoda tohoto přístupu je jasná. Povinnost hledat všechny důsledky teorie může řádově zpomalit operaci \icode{SetTrue} pro řešič teorie. Nicméně se ukázalo~\cite{Nieuwenhuis05}, že alespoň v~případě diferenční logiky může tento přístup vést k~rychlé implementaci schopné předčit ostatní alternativy.

\subsection{Návrh řešiče pro diferenční logiku}

\Solver diferenční logiky využívá principy, se kterými jsme se podrobněji seznámili v~předchozích sekcích. Můžeme například předpokládat, že všechna omezení jsou tvaru $x-y \leq c$ (převod z~dalších forem jsme popsali v~sekci \ref{stp}). Využijeme též převodu omezení do tvaru omezujícího grafu, jak bylo uvedeno v~sekci \ref{graf}.

\subsubsection*{Inicializace}
Během inicializace načte řešič vstupní formuli, uloží si všechna rozdílová omezení, která se v~ní vyskytují, a~předá ji DPLL($X$) jako čistě booleovskou formuli. Během tohoto procesu by měl detekovat vztahy mezi literály a~jejich negacemi a~explicitně je poznamenat. Pokud se například na vstupu vyskytnou nerovnice $x-y \leq 1$ a~$y-x \leq -2$, v~oboru celých čísel je jedna negací druhé. Řešič by tak měl abstrahovat tyto výskyty jako $p$ a~$\neg p$ pro booleovskou proměnnou $p$. Ukládá si přitom překladovou tabulku pamatující si pro každou nerovnici abstrahovaný literál, kterému odpovídá. Zároveň s~tím si udržuje pro každou proměnnou seznam všech nerovnic, ve kterých se tato proměnná vyskytuje (účel těchto seznamů je objasněn níže).

\subsubsection*{Ohodnocení literálu}
Jakmile je následně nastavena pravdivostní hodnota některého literálu, převede jej řešič do formy $x-y \leq c$ a~přidá odpovídající hranu do omezujícího grafu. Následně musí objevit všechny důsledky tohoto ohodnocení. Pro jejich nalezení je potřeba zkontrolovat všechny cesty $$x_i \xrightarrow{c_i*} x \xrightarrow{c} y \xrightarrow{c'_j*} y_j$$ a~zjistit, zda nějaké omezení neplyne z~$x_i - y_j \leq (c_i + c + c'_j)$. To budou právě nerovnice tvaru $x_i - y_j \leq c'$, kde $c' \geq c_i + c + c'_j$. 

Abychom prošli všechny tyto cesty procházející nově přidanou hranou, musíme nejdřív nalézt seznam všech vrcholů $x_i$, ze kterých je dosažitelné $x$, a~seznam všech vrcholů $y_j$, které jsou dosažitelné z~$y$. Omezující graf je tedy reprezentován jako oboustranný seznam sousedů. Potom už můžeme spustit obyčejný algoritmus na hledání nejkratší cesty, kterým získáme všechna vhodná $x_i$ společně s~jejich $c_i$, resp. $y_j$ s~jejich $c'_j$. Autoři \cite{Nieuwenhuis05} pro toto prohledávání doporučují následující postup.

Použijeme běžné prohledávání do hloubky. V~něm si označíme každý vrchol, který jsme navštívili poprvé, společně s~jeho vzdáleností. Navštívíme-li pak již objevený vrchol znovu, pokračujeme v~prohledávání pouze tehdy, když je jeho současná vzdálenost menší než naše uložená vzdálenost. Zároveň si všechny objevené $x_i$ a~$y_j$ ukládáme do dvou odlišných seznamů. Po skončení prohledávání spočteme pro oba seznamy počet všech nerovnic, ve kterých se proměnné v~seznamu vyskytují (tyto nerovnice si pro každou proměnnou pamatujeme z~inicializace).

Vyskytují-li se potom například $x_i$ celkově v~menším počtu proměnných, projdeme pro každé $x_i$ seznam všech nerovnic, ve kterých se vyskytuje, a~zkontrolujeme, zda se nejedná o~důsledek nalezené cesty z~$x_i$ do nějakého $y_j$.

\subsubsection*{Hledání příčin}

Jak víme z~\ref{smt}, řešiče teorie musí implementovat operaci \icode{Explain}, která pro nalezený důsledek vrátí množinu jeho příčin. Tato operace je důležitá pro budování implikačního grafu v~DPLL($X$) a~určení vhodné úrovně backtrackingu při nalezení sporu. Řešič DL popsaný v~\cite{Nieuwenhuis05} tuto operaci provádějí následovně.

Když je do omezujícího grafu přidána $n$-tá hrana, zapamatujeme si k~této hraně její odpovídající $n$. U~nalezených důsledků si pamatujeme $n$ hrany, jejímž přidáním důsledek vznikl. Když pak hledáme vysvětlení hrany $h$ tvaru $x-y \leq c$, spustíme prohledávání z~$x$ do $y$ stejným způsobem jako po přidání hrany. Prohledávání ale pouštíme jen do délky nejvýše $c$ a~jen přes hrany s~číslem vložení nejvýše $n$. Tato omezení zvyšují efektivitu (zmenšujeme prostor k~prohledání) a~zaručují, že nepoužíváme hrany přidané až po důsledku (což by působilo problémy v~implikačním grafu DPLL($X$)). Nalezená cesta tak má délku kratší nebo rovnou $c$ a~sestává se jen z~dříve přidaných hran, z~čehož jasně vidíme, že $h$ je důsledek hran na této cestě.

\section{Úpravy referenčního algoritmu}\label{upravy}

Naše implementace je převážně založená na algoritmu tak, jak ho postulovali Nieuwenhuis a~Oliveras \cite{Nieuwenhuis05}. Přesto se liší v~některých implementačních detailech, daných zejména odlišnostmi OpenSMT od referenčního DPLL($T$) frameworku. %% TODO finish

Prvním důležitým rozdílem je způsob abstrakce literálů teorie. V~OpenSMT neexistuje způsob, kterým bychom mohli informovat hlavní engine o~vztazích mezi různými nerovnicemi. Obsahuje-li například vstupní formule nerovnice $(x-y\leq 3)$ a~$(y-x<-3)$, algoritmus popsaný v~\cite{Nieuwenhuis05} je pro DPLL($X$) abstrahuje do booleovských symbolů $p$ a~$\neg p$. Této abstrakce nejsme pomocí rozhranní v~OpenSMT schopni. Náš řešič si tedy musí tyto vztahy udržovat interně. 

S~tímto omezením úzce souvisí druhý zásadní rozdíl. Můžeme si všimnout, že schéma uvedené v~sekci \ref{smt} umožňuje frameworku ohodnotit literál pouze jako \emph{true}. Uvědomme si, že to obecně není nijak omezující. Ohodnocení termu $p$ jako \emph{false} totiž můžeme snadno zařídit voláním \icode{SetTrue($\neg p$)}. Protože však OpenSMT nezná tato mapování mezi termy a~jejich negacemi, není pro nás tento přístup validní. Namísto toho využíváme přímočařejší implementace, kde termům můžeme přiřadit ohodnocení \emph{true} i~\emph{false}. Tato odlišnost vyžaduje několik modifikací našeho řešiče. 

Předně musíme být schopni detekovat, zda jedna nerovnice neodpovídá negaci druhé, jak už jsme uvedli výše. Jakmile jsme toho schopni, můžeme záporné ohodnocení hrany implementovat jako přidání negace této hrany do omezujícího grafu. Jak ale postupovat, pokud jsme u~některé hrany tuto negaci nenašli? Naivní přístup by byl vytvořit negaci takové hrany ve chvíli, kdy se objeví její záporné ohodnocení. Takové řešení však není možné: uvažujme neohodnocenou hranu $h$, která ve vstupní formuli nemá svou negaci. Mějme dále v~omezujícím grafu nějakou množinu hran $M$ takovou, že $M \cup \{h\}$ tvoří záporný cyklus. Pokud $M$ existuje, očividně je $\neg h$ důsledkem tohoto grafu. Jelikož se ale $\neg h$ nevyskytuje ve vstupní formuli a~$h$ nebyla nikdy záporně ohodnocená, nevyskytuje se $\neg h$ v~seznamu možných hran našeho řešiče. Algoritmus hledání důsledků ji tudíž nemůže nalézt. Kladným ohodnocením $h$ potom vytvoříme záporný cyklus v~omezujícím grafu, čímž porušíme invariant našeho algoritmu a~nekonečně zacyklíme příští hledání důsledků.

Jak vidíme, negace všech hran musí být známy předtím, než proběhne prohledávání grafu. Tento problém jsme se tedy rozhodli vyřešit už při oznamování možných literálů. Když je řešiči předán literál vyskytující se ve formuli, vytvoříme nejen hranu odpovídající tomuto literálu, ale okamžitě i~hranu odpovídající její negaci. Při předávání dalších literálů je pak jen třeba ověřit, zda neodpovídají některé již vytvořené hraně. Podrobněji tento postup popisujeme v~sekci \ref{add}.

Za zmínku stojí také to, že OpenSMT nevyžaduje nutně kontrolu konzistence po každém ohodnocení. Tuto možnost ponechává řešičům pouze volitelně a~přidává do rozhranní operaci \icode{Check}, pomocí které kontrolu explicitně vyvolává. Může tím sice způsobit, že nějakou dobu pokračuje s~hodnocením proměnných v~nekonzistentním stavu, ale na druhou stranu tím lze dosáhnout vyšší efektivity, pokud je v~dané teorii kontrola náročnou operací. Jelikož referenčním algoritmem dokážeme triviálně detekovat sporná ohodnocení (\ref{rozhod}), náš řešič o~nich informuje vždy již při jejich oznámení. Z~\icode{Check} se pak stává prázdná operace, která jen vrací současný stav řešiče.

Posledním větším rozdílem je způsob zpracování konfliktů. Jedním z~důsledků vyčerpávající propagace je fakt, že řešič nemusí kontrolovat nesplnitelnost předaného ohodnocení. Zapříčinilo-li by přidání nějaké hrany spor, negace této hrany je důsledkem omezujícího grafu. Můžeme přitom předpokládat, že DPLL($X$) negaci objeveného důsledku nikdy řešiči nepředá. OpenSMT se v~tomto ohledu liší ve způsobu, jakým analyzuje sporný stav. Jeho řešiče obecně nepodporují operaci \icode{Explain}, vracející pro nějaký důsledek množinu jeho příčin. Místo toho implementují funkci \icode{getConflict}, která hledá nesplnitelnou množinu literálů. Pro použití této funkce, nutné k~určení úrovně backtrackingu, se ale nejdřív musí řešič dostat do nekonzistentního stavu.

Tyto dva přístupy jsou naštěstí ekvivalentní. Když OpenSMT objeví spor, předá řešiči nějaké zaručeně sporné ohodnocení, čímž ho dostane do nekonzistentního stavu. Z~tohoto důvodu musíme před každým přidáním hrany kontrolovat, zda není sporná se stávajícím ohodnocením (více viz.~\ref{add}). Náš řešič si zapamatuje hranu $h$ odpovídající tomuto ohodnocení a~funkce \icode{getConflict} pak odpovídá volání \icode{Explain($\neg h$)}. 

\section{Popis běhu programu}

\begin{figure}
	\centering
	\includegraphics[width=\textwidth]{interface}
	\caption{Ukázka interakce řešiče s~OpenSMT (pseudokód)}
\end{figure}

V~první fázi algoritmu jsou řešiči předány všechny literály teorie, které se vyskytují ve vstupní formuli. Řešič z~nich nejprve extrahuje relevantní hodnoty. Následně zkontroluje, zda se nejedná o~negaci některého z~již zapamatovaných literálů. Pokud ano, pouze tuto negaci explicitně označí. V~opačném případě vytvoří hranu odpovídající tomuto literálu a~zároveň a~priori hranu tvořící negaci tohoto literálu, jak je popsáno v~sekci \ref{upravy}. Obě nové hrany se stanou součástí úložiště a~jsou zařazeny do překladových tabulek. Zbytek výpočtu pak již může předpokládat, že pracujeme pouze se známými literály, pro něž máme vytvořené hrany.

Poté, co jsou všechna omezení načtena, nastává hlavní část programu. Během té postupně řešič dostává rozhodnutá ohodnocení literálů. Když nějaké obdrží, přidá do omezujícího grafu hranu odpovídající tomuto ohodnocení. Následně proběhne prohledávání objevující všechny důsledky tohoto ohodnocení. Tyto důsledky jsou přeloženy zpět do vhodné formy a~oznámeny frameworku.

Po nějaké sekvenci těchto ohodnocení buď nalezneme splňující ohodnocení celé formule, nebo se dostaneme do sporu. Spor můžeme rozpoznat tak, že se chystáme přidat do grafu hranu, jejíž negace byla buď dříve do grafu přidána, nebo byla nalezena jako důsledek dřívějšího ohodnocení. V~takovém případě řešič oznámí selhání tohoto ohodnocení a~přejde do chybového stavu. Jakmile je řešič v~chybovém stavu, automaticky zamítá všechna nová ohodnocení. V~této situaci následuje nalezení nesplnitelné množiny. Pokud byla objevená negace hrany explicitně přidána do grafu, je tato množina triviální. Jinak ji získáme modifikovanou verzí prohledání grafu (viz.~\ref{alg}). Jakmile je nesplnitelná množina nalezena a~předána frameworku, rozhodne se podle ní úroveň backtrackingu. Ten je v~OpenSMT řešen obecně pomocí systému záchytných bodů. Řešič může být v~libovolnou chvíli požádán, aby uložil svůj aktuální stav na zásobník. V~případě nutnosti je mu pak sděleno, aby odebral několik bodů z~vrchu tohoto zásobníku a~tím se vrátil do dřívějšího stavu. 

Jakmile se řešič vrátí do konzistentního stavu, tento proces se opakuje, dokud není nalezeno nějaké splnitelné ohodnocení, nebo dokud framework nevyčerpá všechny možnosti ohodnocení. Splnitelnost formule je určena tím, který z~těchto dvou případů nastane. V~případě, že je formule splnitelná, může být řešič na závěr výpočtu ještě požádán, aby vytvořil její model, tedy nalezl konzistentní hodnoty pro všechny proměnné obsažené v~literálech teorie.

\section{Srovnání reálné a~celočíselné verze} \label{int_v_real}

Algoritmus uvedený v~sekci \ref{alg} je s~drobnými úpravami použitelný jak pro RDL, tak pro IDL. Přestože v~této práci implementujeme pouze celočíselnou verzi problému, přišlo nám názorné zamyslet se nad rozdíly mezi oběma variantami.

%Uveďme nejprve pro zajímavost, že pokud na vstupu podporujeme i~rovnice a~jejich negace, ověření konzistence je možné v~reálných číslech provést polynomiálně, zatímco pro celožíselný obor je ověření NP-těžké --- jsme schopni na něj převést např. problém $k$-barevnosti grafu \cite{slides}.

Prvním zjevným rozdílem pro potřeby naší implementace je reprezentace čísel. Reálná čísla je potřeba reprezentovat jinak, než čísla celá. OpenSMT definuje vlastní datový typ pro reálná čísla nazvaný \icode{FastRational}. V~různých částech kódu jej můžeme najít také pod aliasy \icode{Real} nebo \icode{Number}. Tento typ má několik výhod oproti běžným primitivním typům s~pohyblivou desetinnou čárkou. Především se jedná o~typ s~teoreticky neomezenou velikostí. Jelikož využívá struktur větších než jedno procesorové slovo, není limitován kapacitami procesoru. Důsledkem toho je to také typ s~libovolnou přesností. Netrpí tak zaokrouhlovacími chybami a~ztrátou platných číslic u~velkých hodnot jako např. \icode{float} a~\icode{double}. Tyto vlastnosti jsou důležité, jelikož sémantika SMT-LIB\footnote{sada standardů a~knihoven zabývající se SMT řešiči, vůči níž je OpenSMT implementováno} vyžaduje, aby všechny číselné výpočty probíhaly s~neomezenou přesností.

Samozřejmě bychom mohli \icode{FastRational} použít i~v~celočíselném řešení. Usnadnilo by nám to návrh datových struktur a~umožnilo větší znovupoužitelnost kódu. Testováním se však ukázalo, že aritmetické operace na \icode{FastRational} jsou citelně pomalejší než u~primitivních typů. Náš projekt tak používá primitivní celočíselný typ, konkrétně typ \icode{ptrdiff\_t}\footnote{\icode{ptrdiff\_t} byl vybrán jako znaménkový typ s~největším rozsahem, který můžeme na cílové architektuře očekávat}. Ten bohužel podporuje pouze hodnoty omezeného rozsahu a~je tak v~rozporu se standardem. Protože ale v~OpenSMT zatím neexistuje ekvivalent \icode{FastRational} pro celá čísla, uznali jsme jej jako nejlepší volbu. Toto rozhodnutí bylo podpořeno skutečností, že omezení rozsahu se experimentálně ukázalo jako zanedbatelné pro běžné použití --- z~661 testů knihovny SMT-LIB náš řešič vrátil ve všech případech správný výsledek (podrobněji viz.~kapitola \ref{experiment}).

Zásadní algoritmický rozdíl je také v~tvorbě negací. Máme-li například na vstupu nerovnici $(x-y<k)$, v~IDL ji triviálně převedeme do tvaru $(x-y\leq k-1)$. V~RDL se ovšem ostrých nerovností tak snadno nezbavíme. Nieuwenhuis a~Olivieras \cite{Nieuwenhuis05} navrhují pro tento případ postup, který využili např. Dutertre a~de Moura ve svém řešiči pro lineární aritmetiku \cite{Dutertre06}. Ten je založen na následujícím tvrzení.
\begin{tvrz}[Dutertre a~de Moura \cite{Dutertre06}, Lemma 1] 
	Nechť množina literálů lineární aritmetiky $\Gamma$ obsahuje ostré nerovnosti $S = \{p_1 > 0,\dots,p_n > 0\}$. $\Gamma$ je splnitelná právě tehdy, když existuje racionální $\delta > 0$ takové, že $\Gamma_\delta = (\Gamma \cup S_\delta) \setminus S$ je splnitelná, kde $S_\delta = \{p_1 \geq\delta,\dots,p_n \geq\delta\}$.
\end{tvrz}

Toto tvrzení umožňuje nahradit všechny ostré nerovnice neostrými, známe-li dostatečně malé $\delta$. Hodnotu $\delta$ přitom Dutertre a~de Moura nepočítají přímo, ale používají ho pouze symbolicky jako \emph{infinitesimální parametr}. Omezení a~proměnné pak nejsou ohodnoceny běžnými racionálními čísly, ale dvojicemi čísel $(c, k)$, reprezentujícími výraz $c + k\delta$, na kterých zavedeme odpovídající aritmetické a~porovnávací operace. Vhodnou substituci za $\delta$ je po této substituci třeba hledat až na samotném konci výpočtu, pokud chceme najít model spňujícího ohodnocení. V~kontextu OpenSMT už je tento přístup použit rovněž v~řešiči teorie lineární aritmetiky.

Na závěr ještě připomeňme omezení týkající se datových struktur použitých v~OpenSMT. Jelikož \icode{FastRational} obsahuje ukazatele, nelze jej --- ani typy, které ho obsahují --- použít například jako prvek třídy \icode{vec}. Na tato omezení je třeba dbát při přechodu z~celočíselné verze na reálnou.

\chapter{Vývojová dokumentace}

%Tato kapitola detailně popisuje implementaci našeho programu. V každé sekci podrobně rozebereme jednu část řešení, rozhodnutí vedoucí k její implementaci a případné problémy, na které jsme narazili během vývoje.
\section{Datové struktury}\label{data}

Základní datovou strukturou v~OpenSMT je \icode{Pterm}, reprezentující jeden term vyskytující se ve vstupní formuli. Odkazy na tyto termy jsou pak předávány pomocí referenční struktury \icode{PTRef}. Ta obsahuje pouze numerický identifikátor použitý k~jejímu rozlišení. Mapování jednotlivých referencí na odpovídající termy přitom zařizuje třída \icode{Logic}, respektive její potomci. \icode{Logic} si ukládá převodní tabulku párující \icode{PTRef} a~jejich odpovídající \icode{Pterm} a~poskytuje také mechanismy pro vytváření nových termů. Její potomci pak rozšiřují tyto mechanismy o~možnosti odpovídající dané teorii, např. s~\icode{LALogic} jsme schopni vytvářet termy odpovídající nerovnicím z~lineární aritmetiky. Jelikož rozdílová omezení jsou speficickým tvarem lineárních nerovnic, využívá náš řešič právě schopností \icode{LALogic}, konkrétně \icode{LIALogic} pro implementovanou celočíselnou verzi.

Ohodnocení proměnných je uloženo ve struktuře \icode{PtAsgn}, která obsahuje \icode{PTRef} odpovídající ohodnocené proměnné a~\icode{lbool} označující její ohodnocení (\icode{lbool} je běžný optional boolean).

Samotné termy mají stromovou strukturu. Pokud se nejedná o~atomickou proměnnou, reprezentuje term nějaký $n$-ární funkční či relační symbol společně s~jeho argumenty. K~rozlišení typu symbolu nám opět poslouží API třídy \icode{Logic}, pro přístup k~argumentům je pak použit operátor \icode{[]}. Reprezentuje-li například \icode{Pterm~p} term $(x \lor y)$, budeme mít přístup k~proměnným \icode{p[0]} a~\icode{p[1]}, což jsou \icode{PTRef} reprezentující $x$, respektive $y$. 

\begin{code}[label=Příklad práce s~termem p $\approx (4 \leq x)$]
PTRef p;    
assert( logic.isNumLeq(p) );
Pterm &term = logic.getPterm(p);

PTRef c = term[0];
assert( logic.isNumConst(c) );
opensmt::Number n = logic.getNumConst(c);
assert( n == 4 );

PTRef x = term[1];
assert( logic.isNumVar(x) );
\end{code}

Nejdůležitější datovou strukturou našeho řešiče je \icode{Edge}. V~té jsou uloženy informace o~jedné hraně omezujícího grafu. Konkrétně tedy obsahuje reference na její vstupní a~výstupní vrchol, ohodnocení, odkaz na svou negaci a~informaci o~tom, kdy byla přidána do omezujícího grafu. Po vzoru frameworku přitom zbytek řešiče nepracuje přímo s~těmito strukturami, ale s~jejich referencemi \icode{EdgeRef}, které opět obsahují pouze jednoznačný numerický identifikátor. Samotné hrany se pak nacházejí jen v~centrálním úložišti, které tvoří třída \icode{STPStore}. Ta zařizuje zejména tvorbu nových hran a~převod z~\icode{EdgeRef} na \icode{Edge\&}. Použit je i~protějšek k~\icode{EdgeRef} pro vrcholy, struktura \icode{VertexRef}. Jelikož se však s~vrcholy nepojí žádná informace, nejedná se o~odkaz na další strukturu, ale pouze o~symbolické reference, sloužící pro vzájemné rozlišení jednotlivých vrcholů.

%\captionof{verbatim}{test}
\begin{code}[label=Deklarace struktury Edge]
struct Edge {
    VertexRef from, to;    
    EdgeRef neg;           
    ptrdiff_t cost;
    uint32_t setTime;
}
\end{code}

Jelikož náš řešič dostává od frameworku informace o~proměnných zásadně jako \icode{PTRef}, potřebujeme způsob, jak přecházet mezi reprezentací frameworku a~interní reprezentací našeho řešiče. K~tomuto účelu slouží třída \icode{STPMapper}. V~této třídě se vyskytuje hned několik druhů převodních tabulek. Pamatuje si převod z~\icode{PTRef} na \icode{VertexRef}, přiřazující proměnné k~vrcholům grafu, a~převod z~\icode{PTRef} na \icode{EdgeRef}, přiřazující nerovnice k~hranám. Tyto převody jsou zásadní pro interpretaci příkazů frameworku. Pro hrany si pamatuje i~opačný převod, mapující \icode{EdgeRef} zpět na odpovídající \icode{PTRef}. Ten je důležitý pro oznámení nalezených dedukcí (viz.~\ref{dusl}). Pro účely oznámení nesplnitelné množiny literálů si pro hrany právě v~grafu pamatujeme i~mapu z~\icode{EdgeRef} na \icode{PtAsgn}, které způsobily jejich přidání do grafu. Uvědomme si, že tento převod není zaměnitelný s~předchozím převodem na \icode{PTRef}, jelikož hrana se může vyskytnout v~grafu z~důvodu záporného ohodnocení její negace. \icode{STPMapper} si navíc pamatuje pro každý vrchol seznam všech hran, ve kterých se daný vrchol vyskytuje, jak je popsáno v~\ref{alg}.

Samotný omezující graf ukládáme do struktury \icode{STPEdgeGraph}. Ta obsahuje seznam přidaných hran a~oboustranný seznam sousedů pro všechny vrcholy grafu. S~grafem přímo manipuluje třída \icode{STPGraphManager}, která působí jako hlavní výpočetní třída řešiče. Provádí přidávání hran do grafu a~jejich případné odebírání z~grafu, ale i~hledání důsledků přidané hrany a~hledání vysvětlení nalezeného důsledku.

Třídu \icode{STPModel} využijeme, pokud chceme pro splnitelnou množinu nerovnic najít nějaké ohodnocení proměnných. \icode{STPModel} dostane kopii grafu, ze které vytvoří mapu ohodnocení obsažených vrcholů.

Všechny struktury řešiče spojuje dohromady hlavní třída \icode{STPSolver}. Jakožto potomek \icode{TSolver} implementuje tato třída rozhranní mezi řešičem a~zbytkem frameworku. %% TODO add more.

\begin{figure}
	\centering
	\includegraphics[width=0.8\textwidth]{class_deps}
	\caption{Závislosti mezi strukturami řešiče}
\end{figure}

Je vhodné zmínit, že za dobu vývoje OpenSMT v~něm vzniklo několik implementací základních datových struktur. Hojně užívaným příkladem je třída \icode{vec}, reprezentující běžný vektor. Vyjma malých rozdílů API a~údajné vyšší efektivity na primitivních typech se tyto třídy výrazně neliší od implementací ze standardní knihovny. Konkrétně třída \icode{vec} je navíc omezena skutečností, že jejími prvky nemohou být typy obsahující odkazy (vyjímkou z~tohoto pravidla je zvlášť implementovaný \icode{vec$\langle$vec$\langle$T$\rangle\rangle$}). Za účelem konzistence se zbytkem frameworku jsme se přesto rozhodli využívat tyto lokální struktury všude, kde je to možné. 
\newpage
\section{Přidávání literálů}\label{add}

Na samotném začátku výpočtu se OpenSMT postupně pro každý term zeptá řešiče, zda se jedná o~literál jeho teorie. Jelikož tento proces slouží primárně k~jejich odlišení běžných booleovských termů, nezkoumáme do hloubky struktury termu, ale provádíme jen povrchovou kontrolu. Pokud předpokládáme, že framework použije náš řešič jen pro odpovídající teorii, je taková kontrola dostačující.
\begin{code}
bool STPSolver::isValid(PTRef tr) { return logic.isNumLeq(tr); }
\end{code}

Jakmile jsou termy takto rozlišeny, řešič je informován o~všech literálech, jejichž ohodnocení mu může být oznámeno. K~tomu je využita funkce \icode{declareAtom(PTRef)}. V~té už musíme projít strukturu literálu, abychom z~něj extrahovali relevantní informace. Důležité pro nás přitom je, že díky způsobu, kterým OpenSMT vytváří své termy, nemusíme provádět konverze z~různých tvarů omezení (popsaných v~sekci \ref{stp}). Neostrých nerovností a~obrácených znamének jsme tak v~literálech zbaveni dříve, než se o~nich řešič vůbec dozví --- všechny literály naší teorie jsou reprezentovány standardní formou $c \leq x - y$, popř. $c \leq \pm x$ pro omezení s~jednou proměnnou.\footnote{Tato kanonická transformace mimo jiné ospravedlňuje výše zmíněnou implementaci \icode{isValid}.} Uveďme pro úplnost, že rozdíl je v~této reprezentaci nahrazen součtem záporu. Přesnější popis struktury literálu tak je spíše $c \leq x + (-1 \cdot y)$.

Z~této formy nám už nedělá problém určit hodnotu $c$ a~reference na $x$ a~$y$ postupem naznačeným v~předchozí sekci. Musíme přitom dbát pouze na to, že nerovnice je v~opačné formě než té námi používané. Hrany omezujícího grafu tudíž povedou z~$y$ do $x$ s~ohodnocením $-c$. Tuto odlišnost zdůrazňujeme, protože je v~implementaci práce použití identifikátorů \icode{x} a~\icode{y} standardizováno ve smyslu cílového, resp. zdrojového vrcholu hrany.

\begin{figure}
	\centering
	\includegraphics[width=\textwidth]{ptref_structure}
	\caption{Ukázka struktury termu (\icode{PTRef tr} $\approx c \leq x - y$)}
\end{figure}

Jakmile jsme získali informace o~hraně, musíme nejprve zkontrolovat, zda už hrana v~našem řešiči neexistuje. To se může stát, pokud je hrana negací jiné již přidané hrany (viz. níže). %% TODO garbage reference
Hledání hrany probíhá postupným projitím seznamu všech hran, ve kterých se vyskytuje $y$ (seznamy si pro každý vrchol ukládá \icode{STPMapper}). Probíhá tedy lineárně v~počtu hran obsahujících $y$. Jelikož ale přidávání literálů celkově zabírá jen minimální část celého výpočtu, usoudili jsme, že lineární složitost této operace je pro nás přijatelným kompromisem.

Objevíme-li hranu odpovídající přidanému literálu, stačí nám přidat do třídy \icode{STPMapper} tuto asociaci. Všechny ostatní informace o~hraně už známe. V~opačném případě musí ze všeho nejdříve \icode{STPStore} tuto hranu vytvořit (případně s~vrcholy, které jsme ještě nepotkali). Jakmile je hrana vytvořená, vytvoříme předběžně i~její negaci (z~důvodů popsaných v~sekci \ref{upravy}). Obě nově vytvořené hrany pak přidáme do seznamů hran pro jejich vrcholy. U~původní hrany si navíc uložíme její asociaci s~oznámeným literálem (pro negaci zatím tato asociace neexistuje). 

Poté, co tento proces proběhne pro všechny literály ve vstupní formuli, jsme schopni každý literál převézt na odpovídající hranu a~naopak a~dokážeme reprezentovat libovolné ohodnocení přidáním odpovídající hrany do omezujícího grafu.

\section{Oznámení ohodnocení a~hledání důsledků}\label{dusl}

Nejdůležitější část výpočtu se odehrává ve funkci \icode{STPSolver::assertLit}, pomocí které se řešič dozvídá o~nových ohodnocení literálů. Funkce nejprve převede literál na odpovídající hranu. Následně zkontroluje, zda jsme hranu již neobjevili. Je-li hrana označená jako \emph{pravdivá} (tj. je buď v~omezujícím grafu, nebo v~seznamu důsledků), můžeme proces ukončit a~oznámit úspěšné přidání ohodnocení. Je-li naopak pravdivě označená negace hrany, přidané ohodnocení je sporné. V~takovém případě si zapamatujeme bod, ve kterém jsme se do sporu dostali, a~skončíme proces oznámením o~neúspěchu přidání hrany. 

Pokud nenastane ani jedna z~těchto možností, uložíme si asociaci hrany s~danou proměnnou \icode{PtAsgn}, přidáme ji do omezujícího grafu a~začneme hledat důsledky tohoto přidání. Přidání hrany do omezujícího grafu znamená její přidání do seznamu hran grafu a~do obou seznamů sousedů a~nastavení její vlastnosti \icode{setTime} na počet aktuálně přidaných hran. Kontrolu pravdivosti hrany z~předchozího odstavce provádíme pomocí vlastnosti \icode{setTime}, která je pro pravdivé hrany nenulová.

Hledání důsledků zařizuje funkce \icode{STPGraphManager::findConsequences}. Ta nejprve najde seznam možných počátečních a~koncových vrcholů pro cesty procházející přidanou hranou. K~tomu využíváme prohledávání do hloubky, jak bylo popsané v~referenčním algoritmu. Prohledávání přitom probíhá sekvenčně --- nejprve hledáme počáteční vrcholy, poté hledáme vrcholy cílové. Jelikož prohledávání na společných datech provádí pouze čtecí operace, bylo by možné tento proces paralelizovat a~hledat obě množiny současně. Během vývoje jsme zkoušeli uplatnit tento přístup, nicméně režie vytváření a~rušení vláken  se ukázala časově náročnější než samotné prohledávání. Prozkoumali jsme také možnost nastavit hranici na velikost grafu určující, zda výpočet proběhne sekvenčně, či paralelně. V~praxi se ale ukázalo, že hranice byla buď příliš vysoká a~vícevláknový přístup nebyl nikdy použit, nebo byla příliš nízká a~použití vláken stále zhoršovalo výkonnost programu. Rozhodli jsme se tak zůstat u~běžné jednovláknové varianty. Využití perzistentního pole vláken nebo rigoróznější hledání limitu pro použití vlákna jsou možné oblasti dalšího zkoumání, dle našeho úsudku by ale nepřinesly podstatné zrychlení algoritmu.

Poté, co obě množiny nalezneme, vybereme tu, jejíž vrcholy se celkově objevují v~menším množství hran (tyto součty počítáme v~průběhu DFS). Pro kažý vrchol z~vybrané množiny projdeme seznam všech hran, v~nichž se vyskytuje (tyto seznamy si ukládá \icode{STPMapper}) a~uložíme si všechny, které jsou důsledky naposledy přidané hrany. Aby byla hrana $a \xrightarrow{c'} b$ důsledkem právě přidané hrany $y \xrightarrow{c} x$, musí platit 
\begin{enumerate}
	\item Existuje cesta $a \rightarrow y \xrightarrow{c} x \rightarrow b$.
	\item $c'$ je větší než délka nekratší takové cesty.
\end{enumerate}

Snadno nahlédneme, že tyto dvě podmínky pokrývají právě všechna omezení, která jsou důsledky přidání hrany $y \xrightarrow{c} x$ do grafu. Jakmile objevíme všechny takovéto hrany, musíme je převézt zpět do tvaru \icode{PtAsgn} a~předat je frameworku. Odpovídá-li přitom hraně \icode{h} literál \icode{tr} a~negaci \icode{h} literál \icode{nr}, objevení \icode{h} jakožto důsledku způsobí označení \icode{PtAsgn(tr,true)} a~\icode{PtAsgn(nr,false)} jako důsledků oznámeného ohodnocení (pokud \icode{tr} a~\icode{nr} existuje). Pokud existuje odpovídající literál \icode{nr} i~k~negaci hrany právě přidané do grafu, nesmíme zapomenout ani na vytvoření důsledku \icode{PtAsgn(nr,false)} pro tuto hranu.

Oznamování takto nalezených dedukcí provádíme pomocí rozhranní abstraktní třídy \icode{TSolver}, jíž je \icode{STPSolver} potomkem. Ta poskytuje funkci \icode{storeDeduction}, které můžeme předat námi vytvořené důsledky. Funkce technicky vzato bere jako parametr strukturu \icode{PtAsgn\_reason}, jež ale v~řešiči existuje pouze z~historických důvodů a~funkčně se nijak neliší od běžného \icode{PtAsgn}. Voláním \icode{storeDeduction} uložíme nalezené důsledky ve vnitřní struktuře \icode{TSolver}, z~níž už je zbytek řešiče dokáže získat.
\section{Rozhodování o~splnitelnosti}\label{rozhod}



\section{Hledání konfliktů a~backtracking}

\section{Nalezení splňujícího ohodnocení}

\chapter{Experimentální měření}

\section{Metodologie}

\section{Výsledky}

\begin{table}[h]
	\centering
	\begin{tabular}{l@{\hspace{1cm}} D{.}{,}{3.0} D{.}{,}{6.1} D{.}{,}{3.0} D{.}{,}{3.0} D{.}{,}{3.0}}
		\toprule  
		& \mc{\textbf{Úspěšně}} & \mc{\textbf{Reálný}} & \mc{\textbf{Vyřešeno}} & \mc{\textbf{Vyřešeno}} & \mc{} \\
		\pulrad{\textbf{Řešič}} &\mc{\textbf{vyřešeno}} & \mc{\textbf{čas}} & \mc{\textbf{SAT}} & \mc{\textbf{UNSAT}} & \mc{\pulrad{\textbf{Timeout}}}\\
		\midrule
		Yices & 740 & 128683.6 & 484 & 256 & 94 \\
		z3 & 727 & 124296.1 & 485 & 242 & 82 \\
		CVC4 & 682 & 231953.6 & 425 & 257 & 152\\
		\textbf{OpenSMT} & 661 & 260896.8 & 412 & 249 & 173 \\
		veriT & 621 & 308952.9 & 372 & 249 & 213\\
		MathSAT & 585 & 337559.4 & 341 & 244 & 249\\
		SMTInterpol & 546 & 396759.1 & 313 & 233 & 288\\
		\bottomrule
	\end{tabular}
	\caption{Srovnání SMT řešičů}
\end{table}

\section{Srovnání}


\chapter*{Závěr}
\addcontentsline{toc}{chapter}{Závěr}

V~rámci práce jsme pro OpenSMT vytvořili řešič diferenční logiky nad množinou celých čísel. Seznámili jsme se s~problémem STP, rozebrali jeho vlastnosti a~ukázali jeho transformaci na grafový problém. Zvážili jsme způsoby jeho řešení v~kontextu SMT řešičů a~analyzovali existující algoritmy navržené pro framework DPLL($T$). Popsali jsme metodu vyčerpávající propagace teorie a~její aplikaci pro diferenční logiku.

S~ohledem na tyto poznatky jsme pak implementovali samotný řešič teorie. Naším cílem přitom bylo vytvořit řešič, který bude dobře integrován se zbytkem frameworku a~dosáhne výkonu srovnatelného s~ostatními moderními SMT řešiči. Výsledek naší práce jsme s~těmito řešiči experimentálně srovnali a~ukázali jsme, že se nám podařilo daného cíle dosáhnout --- OpenSMT je srovnatelné či dokonce rychlejší než některé z~nejznámějších SMT řešičů současnosti. Nedosahuje zatím účinnosti těch naprosto nejrychlejších, na jejichž vývoji se soustavně podílí desítky dedikovaných výzkumníků. Věříme ale, že poskytuje dobrý základ, který už v~současné formě nabízí rozumnou alternativu k~současným řešičům a~jehož využitelnost bude růst s~budoucími zlepšeními. 

\subsection*{Možná budoucí rozšíření}

Hlavním úkolem v~nejbližší budoucnosti frameworku je rozšíření našeho řešiče o~podporu problémů nad doménou reálných čísel. S~využitím poznatků uvedených v~sekci~\ref{int_v_real} a~datových struktur již existujících v~OpenSMT máme za to, že by toto rozšíření nemělo být příliš náročné. Většina kódu existujícího v~implementaci celočíselné verze je totiž přenositelná mezi oběma variantami.

Dalším možným směrem budoucího vývoje je práce na zlepšení výkonnosti stávajícího řešiče. Jedním možným směrem této práce by byla analýza implementace použitého algoritmu. Jako příklad uveďme bližší průzkum možnosti použití vícevláknových procesů. Naše zběžné testování tyto přístupy sice zamítlo, ale podrobnější výzkum a~testování by mohly odhalit možnosti ke zrychlení řešiče. Druhou možností pro další výzkum v~tomto směru by pak mohlo být také využití jiných algoritmů v~rámci OpenSMT, ať už by se jednalo o~implementaci již existujících algoritmů, nebo o~vytvoření zcela nových postupů.


%%% Seznam použité literatury
\include{literatura}

%%% Obrázky v bakalářské práci
%%% (pokud jich je malé množství, obvykle není třeba seznam uvádět)
%%% \listoffigures

%%% Tabulky v bakalářské práci (opět nemusí být nutné uvádět)
%%% U matematických prací může být lepší přemístit seznam tabulek na začátek práce.
%%% \listoftables

%%% Použité zkratky v bakalářské práci (opět nemusí být nutné uvádět)
%%% U matematických prací může být lepší přemístit seznam zkratek na začátek práce.
%%% \chapwithtoc{Seznam použitých zkratek}

%%% Přílohy k bakalářské práci, existují-li. Každá příloha musí být alespoň jednou
%%% odkazována z vlastního textu práce. Přílohy se číslují.
%%%
%%% Do tištěné verze se spíše hodí přílohy, které lze číst a prohlížet (dodatečné
%%% tabulky a grafy, různé textové doplňky, ukázky výstupů z počítačových programů,
%%% apod.). Do elektronické verze se hodí přílohy, které budou spíše používány
%%% v elektronické podobě než čteny (zdrojové kódy programů, datové soubory,
%%% interaktivní grafy apod.). Elektronické přílohy se nahrávají do SISu a lze
%%% je také do práce vložit na CD/DVD. Povolené formáty souborů specifikuje
%%% opatření rektora č. 72/2017.
\appendix
\chapter{Přílohy}

\section{Přiložené soubory}\label{content}

Mezi soubory přiloženými k~této práci můžeme nalézt
\begin{itemize}
	\item adresář \icode{opensmt} --- úplný obsah repozitáře OpenSMT, respektive jeho větve určené této práci, k~25. 7. 2020. Soubor \icode{opensmt/README.md} obsahuje mimo jiné instrukce ke~kompilaci programu.
	\item soubor \icode{patch} --- patch oproti větvi master k~25. 7. 2020 obsahující všechny změny provedené v~této práci
	\item soubor \icode{benchmark.ods} --- tabulka s~podrobnými daty o~testech popisovaných v~kapitole \ref{experiment}
	\item soubor \icode{queen-4.smt2} --- ukázkový vstup řešiče, rozebraný v~příloze \ref{priklad}
\end{itemize}

\section{Soubory náležící práci}
V~rámci této práce jsme pracovali se zdrojovým kódem v~souborech
\begin{itemize}
	\tt
	\item src/api/Interpret.cc\footnotemark
	\item src/logics/IDLTheory.cc\footnotemark[\value{footnote}]
	\item src/logics/Theory.h\footnotemark[\value{footnote}]
	\item src/tsolvers/IDLTHandler.\{h|cc\}\footnotemark[\value{footnote}]
	\item src/tsolvers/stpsolver/
	\begin{itemize}
		\item[] STPSolver.\{h|cpp\}\footnotemark[\value{footnote}]
		\item[] STPEdgeGraph.\{h|cpp\}
		\item[] STPMapper.\{h|cpp\}
		\item[] STPModel.\{h|cpp\}
		\item[] STPStore.\{h|cpp\}
	\end{itemize}
	\item test/unit/test\_STPSolver.cpp\footnotemark[\value{footnote}]
\end{itemize}
\footnotetext{Na vytvoření/úpravě souboru se mimo autora práce podílel též Mgr.~Martin Blicha}

\section{Ukázkový příklad vstupu řešiče}\label{priklad}

V~této příloze ilustrativně předvedeme převod problému $n$ dam do formule diferenční logiky a~jeho vyřešení SMT řešičem. Náš ukázkový příklad pracuje s~verzí problému obsahující 4 dámy na šachovnici $4\times 4$.

K~zápisu problému potřebujeme čtyři celočíselné proměnné, $x_0, x_1, x_2, x_3$, určující sloupce obsahující dámu v~daném řádku, a~jednu referenční proměnnou $zero$.
\newpage %% TODO be careful
Na proměnné se pak vztahují tři různé druhy omezení ($i, j \in \{0\dots3\}$):
\begin{itemize}
	\item $0 \leq x_i - zero \leq 3$
	\item $x_i \neq x_j$, respektive $(x_i - x_j \geq 1) \lor (x_j - x_i \geq 1)$
	\item $x_i - x_j \neq \pm(i-j)$.
\end{itemize}

První skupina omezení zaručuje, že dámy umisťujeme na platné sloupce šachovnice. Druhá a~třetí zajišťují, že dvě různé dámy nejsou na stejém sloupci, respektive na stejné diagonále. Chceme-li navíc, aby hodnoty proměnných přesně odpovídaly sloupcům šachovnice, přidáme do formule term $zero = 0$. Převod těchto klauzulí do jazyka SMT-LIB můžeme vidět v~souboru \icode{queen-4.smt2}. Po spuštění řešiče na tomto vstupu obdržíme následující výstup.
\begin{code}
> opensmt queen-4.smt2
sat
(model
(define-fun x0 () Int 2)
(define-fun x1 () Int 0)
(define-fun x2 () Int 3)
(define-fun x3 () Int 1)
(define-fun zero () Int 0)
)
\end{code}
První řádek výstupu udává, že řešič označil formuli jako splnitelnou ($4\times4$ je nejmenší šachovnice, pro kterou existuje řešení). Následuje nalezené ohodnocení všech uvedených proměnných. V~našem případě přitom není těžké ověřit, že je uvedené ohodnocení korektní.

\openright
\end{document}

