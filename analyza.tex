\chapter{Analýza}

\section{Prostředí}

Požadavky na použité prostředí jsou určeny převážně požadavky frameworku OpenSMT, pod nějž tato práce spadá. OpenSMT --- a~tudíž i~tento projekt --- je programován v~jazyce C++, konkrétně ve verzi C++11. Práce byla vyvíjena a~testována na operačním systému s~linuxovým jádrem nad architekturou x64. Jelikož si nejsme vědomi toho, že bychom použili vlastnosti jazyka specifické pro tuto konfiguraci, věříme, že náš kód bude možné bez větších potíží zprovoznit i~na jiných platformách a~operačních systémech.

\section{Fungování SMT řešičů}\label{smt}

Problém splnitelnosti booleovské formule, označovaný též zkratkou SAT, patří k~nejznámějším problémům z~oboru matematické logiky. V~roce 1971 se stal prvním dokazatelně NP-úplným problémem \cite{Cook71} a~v~důsledku toho i~užitečným nástrojem pro teorii složitosti, pomocí něhož lze rozhodovat o~NP-úplnosti dalších problémů.\footnote{Příklady převodů NP-úplných problémů na SAT můžeme nalézt např. v~\cite[kapitola 19]{Mares17}}

V~praktickém použití však SAT naráží na své limitované vyjadřovací schopnosti. Práce s~binárními proměnnými, omezenými pouze na dvě různé hodnoty, může komplikovat převod reálného problému do tvaru booleovské formule. Potřeba užití komplexnějších atomů tak vedla ke zobecnění SAT zvanému \emph{Satisfiability modulo theories} (SMT).

Jak název napovídá, SMT rozšiřuje SAT o~jazyk logických teorií (konkrétně teorií predikátové logiky prvního řádu). máme-li nějakou teorii $t$, pak instancí smt rozumíme formuli jazyka této teorie. jiným pohledem můžeme na instanci smt nahlížet jako na booleovskou formuli, ve které jsme nahradili některé binární proměnné za predikáty obsažené v~$t$. problému vztaženému k~této konkrétní teorii pak říkáme \emph{smt s~ohledem na t}.

jako příklady teorií tradičně řešených v~smt lze uvézt např. teorii lineární aritmetiky (la), teorii neinterpretovaných funkcí s~rovností (euf), či teorii diferenční logiky (dl), kterou se budeme zabývat ve zbytku práce.

vzhledem k~podobnostem mezi sat a~smt není překvapením, že smt řešiče využívají schopností sat řešičů. přechod mezi rozhraním termů teorie a~sat řešičem zpravidla probíhá jedním ze dvou základních způsobů \cite{Nieuwenhuis05}.

první přístup se nazývá \emph{hladový}. hladové smt řešiče operují ve dvou krocích. v~prvním převedou celou vstupní formuli na ekvisplnitelnou booleovskou formuli. druhý krok pak již spočívá jen v~předání této formule existujícímu sat řešiči. pokud bychom tedy pracovali například s~aritmetikou nad osmibitovými čísly, mohli bychom reprezentovat každou proměnnou osmi binárními proměnnými a~aritmetické operace převézt na odpovídající sekvence logických operací.

nespornou výhodou hladových řešičů je možnost použití již existujících metod implementovaných na řešení sat. pro některé teorie se také hladové řešiče ukazují být rychlejší než jejich alternativy. jejich největší problém pak obecně spočívá v~překladu literálů teorie do booleovských formulí. ten musí být zkonstruován samostatně pro každou teorii, a~navíc může v~závislosti na teorii produkovat formule znatelně delší, než byl původní vstup. efektivní převody existují např. pro euf s~omezenou doménou \cite{randal02}, obecně však hladový přístup není příliš rozšířený.

jeho protějškem je takzvaný \emph{líný přístup}. líný přístup se nesnaží měnit strukturu vstupní formule; namísto toho je každý predikát abstrahován pomocí nové binární proměnné. když pak sat řešič rozhodne o~ohodnocení těchto proměnných, oznámí toto rozhodnutí \emph{theory solveru} pro danou teorii. theory solver je schopen určit, zda je dané ohodnocení konzistentní, tzn.~dokáže rozhodnout o~splnitelnosti nějaké konjunkce literálů teorie. sat řešič potom hledá platná ohodnocení, dokud nenajde takové, které theory solver prohlásí za konzistentní.

ve prospěch líných SMT řešičů svědčí fakt, že pro danou teorii často existují dobře známé postupy na ověření konjunkce literálů. Pro LA například můžeme využít metod lineárního programování, theory solver pro DL řeší STP (viz.~\ref{stp}) a~podobně. Mohou však ztrácet efektivitu zejména v~důsledku \emph{slepého prohledávání} \cite{Moura04}, kde hlavní řešič rozhoduje o~hodnotě predikátů, aniž by a~priori věděl o~důsledcích těchto ohodnocení v~rámci teorie, což může vést k~nutnosti vyzkoušet velké množství ohodnocení, než je nalezeno nějaké, které je s~teorií konzistentní.

V~roce 2004 navrhli Gazinger a~kol. nový přístup zvaný \emph{DPLL(T)} \cite{Gazinger04}. DPLL($T$) má koncepčně blíže k~línému vyhodnocování, integruje však těsněji hlavní řešič s~theory solverem. Místo toho, aby využíval theory solveru až po nalezení nějakého ohodnocení, průběžně mu oznamuje dosavadní rozhodnutí a~periodicky se ho ptá na splnitelnost právě dosazené konjunkce. Theory solver pak kromě kontroly splnitelnosti také oznamuje hlavnímu řešiči důsledky této konjunkce. Tím jsme schopni dříve opustit větve rozhodovacího stromu nekonzistentní s~teorií. 

V~jádru tohoto přístupu stojí všeobecný DPLL($X$) engine, využívající DPLL \cite{Davis60} postupu pro SAT řešiče. Tento engine nemusí mít žádné znalosti o~konkrétní teorii. Dosazením theory solveru $Solver_T$ pro danou teorii $T$ za parametr $X$ pak vytvoříme konkrétní instanci DPLL($T$). Hlavní engine komunikuje se $Solver_T$ pomocí následujícího rozhranní \cite{Gazinger04}:

\begin{description}
	\item[Initialize(L: množina literálů).] Inicializuje $Solver_T$ s~$L$ jakožto množinou literálů, které se vyskytují v~problému.
	\item[SetTrue(l: L-literál): množina L-literálů.] Skončí výjimkou, pokud se $l$ ukáže jako nekonzistentní s~dosud zadanými literály teorie. V~opačném případě přidá $l$ do seznamu zadaných literálů a~vrátí množinu L-literálů, které jsou důsledky přidání $l$ do tohoto seznamu.
	\item[IsTrue?(l: L-literál): boolean.] Vrátí \emph{true} právě tehdy, když $l$ je důsledkem seznamu přidaných literálů. \emph{false} tedy vrací, pokud je důsledkem tohoto seznamu $\neg l$, nebo pokud z~něj nevyplývá ani $l$, ani $\neg l$.
	\item[Backtrack(n: přir. číslo).] Odstraní posledních $n$ hodnot ze seznamu zadaných literálů. $n$ nesmí být větší než velikost tohoto seznamu.
	\item[Explain(l: L-literál): množina L-literálů.] Vrátí pokud možno co nejmenší podmnožinu zadaných literálů, z~jejichž konjunkce plyne $l$. Pro $l$ musí platit, že je důsledkem nějaké takové podmnožiny, tedy musí být obsažen v~návratové hodnotě nějakého volání \icode{SetTrue(l')} takového, že $l'$ nebylo zahozeno žádným následným voláním \icode{Backtrack}.
\end{description}

Při použití tohoto rozhranní přitom $Solver_T$ nemusí nic vědět o~implementaci DPLL($X$) enginu. Framework je tedy velice modulární a~snadno rozšiřitelný o~nové teorie. Vyžadujeme pouze, aby byl $Solver_T$ schopný inkrementálně přijímat a~odebírat jednotlivé literály teorie. Tento postup se v~praxi ukazuje jako efektivnější než dostupné alternativy. Většina dnes rozšířených SMT řešičů --- včetně námi používaného OpenSMT2 --- je tedy založena na metodě DPLL($T$).

\section{Rozbor STP}\label{stp}

Jedním ze základních podproblémů vyskytujícím se v~takřka všech plánovacích problémech je takzvaný Simple Temporal Problem (STP). STP poprvé postulovali v~roce 1991 Dechter, Meiri a~Pearl \cite{Dechter91} a~od té doby našel široká využití jak v~informatických oblastech, tak v~oborech od medicíny \cite{Anselma06} po vesmírný let \cite{Fukunaga97}.

Vstupem STP je množina rozdílových omezení, to jest nerovnic tvaru $$x - y \leq c,$$ kde $x$ a~$y$ jsou proměnné a~$c$ je konstanta. V~závislosti na tom, jakou verzi problému řešíme, přitom pracujeme buď s~celočíselnými, nebo s~reálnými hodnotami. Výstupem tohoto problému je pak rozhodnutí, zda existuje ohodnocení proměnných tak, aby byla splněna všechna zadaná omezení. V~rozšíření problému pak můžeme požadovat na výstupu i~nějaké takovéto splnitelné ohodnocení, pokud existuje, případně nalezení pokud možno co nejmenší podmnožiny omezení, která zajišťuje nesplnitelnost problému.

Na první pohled se může zdát pevně daný tvar nerovnic příliš omezující, uvědomme si však, že do této formy můžeme převést několik dalších druhů nerovnic. Nejsnáze zahrneme do problému omezení tvaru $x - y = c$; ty stačí jednoduše nahradit nerovnicemi $x - y \leq c$ a~$x - y \geq c$.

Problematické nejsou ani nerovnice typu $\pm x \leq c$. Pro účely takovýchto omezení si zavedeme novou globální proměnnou $zero$, s~jejíž pomocí převedeme předchozí do tvaru $x - zero \leq c$, respektive $zero - x \leq c$. Pokud pak hledáme splňující ohodnocení proměnných, najdeme takové, kde $zero$ je ohodnoceno nulou. Korektnost tohoto postupu zaručuje následující obecně známé tvrzení.

\begin{tvrz}
	Je-li $\sigma$ splňující ohodnocení nějakého STP a~$\delta$ libovolná konstanta, pak ohodnocení $\pi$ definované pro všechny proměnné $x$ jako $\pi(x) = \sigma(x) + \delta$ je také splňující ohodnocení tohoto STP.
\end{tvrz}
\begin{proof}
	Je-li $\sigma$ splňující ohodnocení, pro libovolné rozdílové omezení $x-y \leq c$ v~problému platí $$\pi(x) - \pi(y) = (\sigma(x) + \delta) - (\sigma(y) + \delta) = \sigma(x) - \sigma(y) \leq c,$$ z~čehož je i~$\pi$ splňující ohodnocení.
\end{proof}

Můžeme do problému zahrnout taktéž omezení tvaru $x - y < c$. Pro celočíselné proměnné lze tuto nerovnici ekvivalentně zapsat jako $x - y \leq c-1$. V~reálné variantě pak nahradíme nerovnici výrazem $x - y \leq c - \delta$, přičemž nenastavujeme okamžitě konkrétní hodnotu $\delta$, ale udržujeme si ji pouze symbolicky a~určujeme její vhodné dosazení až při výpočtu splňujícího ohodnocení. Tento postup je detailněji popsán v~sekci \ref{int_v_real}. Uvědomme si, že pokud jsme schopni vyjádřit ostré nerovnosti, umíme vyjádřit i~negace neostrých nerovností a~naopak.

V~jazyce výrokové logiky pak teorii obsahující výše popsané nerovnice nazveme \emph{teorie diferenční logiky} a~budeme ji značit DL. Celočíselnou variantu této teorie pak budeme značit jako IDL a~reálnou variantu jako RDL. Nahradíme-li pak v~booleovské formuli některé termy těmito nerovnicemi, ověření splnitelnosti takto vzniklé formule je instancí SMT problému s~ohledem na DL.



\section{Převod na grafový problém}\label{graf}

Velkou rozšířenost STP můžeme mimo jiné přisoudit tomu, že jsme schopni ho efektivně řešit. Jelikož se problém skládá výlučně z~lineárních omezení, mohli bychom na první pohled využít metod lineárního programování, jako je například simplexový algoritmus. Tyto metody jsou schopny řešit i~mnohem komplexnější problémy, avšak s~jejich výpočetní silou se pojí znatelně vyšší časová náročnost. Algoritmy specializované na STP se proto už od svého počátku \cite[Kapitola 2]{Dechter91} obracejí jiným směrem, a~to k~formalizmu teorie grafů. Přestože v~průběhu let vzikly různé metody řešení tohoto problému, všechny fungují na základě převedení množiny omezení na takzvaný \emph{omezující graf}.

\begin{definice}[Omezující graf]
	Nechť $\Pi$ je množina rozdílových omezení. Omezujícím grafem této množiny rozumíme hranově ohodnocený orientovaný graf G takový, že vrcholy G tvoří proměnné vyskytující se v~$\Pi$ a~každému omezení $(x-y \leq c) \in \Pi$ odpovídá v~G hrana $\langle x,y\rangle$ s~ohodnocením $c$.
\end{definice}
\begin{pozn}
	Hranu $\langle x,y\rangle$ s~ohodnocením $c$ budeme značit $x \xrightarrow{c} y$. Orientovanou cestu z~$x$ do $y$ se součtem ohodnocení $k$ pak budeme značit $x \xrightarrow{k*} y$.
\end{pozn}

Pro úplnost dodejme, že dvojice proměnných se může vyskytovat v~libovolně mnoha omezeních. Omezující graf je tedy formálně orientovaným multigrafem. Vzhledem k~vzájemné bijekci mezi hranami grafu a~nerovnicemi problému budeme v~průběhu práce volně přecházet mezi oběma reprezentacemi.

Převod do formy grafu je pro řešení problému zásadní. Umožňuje nám totiž formulovat následující klíčové tvrzení.

\begin{tvrz}[Dechter a~kol. \cite{Dechter91}]
	Nechť $\Pi$ je množina rozdílových omezení. Instance STP tvořená touto množinou je splnitelná právě tehdy, když omezující graf $\Pi$ neobsahuje záporné cykly.
\end{tvrz}
\begin{proof}
	Najdeme-li v~omezujícím grafu záporný cyklus obsahující vrchol $x$, sečtením všech nerovnic vyskytujících se v~tomto cyklu dostaneme $x-x \leq c < 0$, z~čehož je jasně problém nesplnitelný. Je-li na druhou stranu problém nesplnitelný, obsahuje $\Pi$ nějakou nerovnici $x - y \leq c$ takovou, že z~$\Pi$ vyplývá $y - x < -c$. Tato implikace znamená, že v~omezujícím grafu existuje cesta $y \xrightarrow{k*} x$ taková, že $k < -c$. Hrana $x \xrightarrow{c} y$ pak společně s~touto cestou tvoří záporný cyklus.
\end{proof}

Hledání splnitelnosti STP jsme tedy schopni převést na hledání záporného cyklu v~grafu. To je problém, který dokážeme efektivně řešit. Využít můžeme např. některý algoritmus na hledání nejkratší cesty, kupříkladu Floydův-Warshallův algoritmus operující v~čase $\Theta(\abs{V}^3)$ nebo Bellmanův-Fordův algoritmus, který má časovou složitost $\Theta(\abs{V}\cdot\abs{E})$.

Tyto algoritmy však trpí pro náš účel zásadním nedostatkem. Jejich použití znamená, že po každém přidání nové hrany do grafu musí znovu proběhnout celé prohledávání. Tento postup není vhodný pro použití v~SMT řešičích, ve kterých je kladen velký důraz na inkrementalitu. V~následující sekci tedy podrobně rozebereme několik postupů pro řešení problémů SMT s~ohledem na DL a~motivujeme výběr námi použitého algoritmu. 

\section{Volba algoritmu}\label{alg}

Jak jsme ukázali v~předchozí sekci, ne všechny algoritmy na rozhodnutí STP jsou dobrou volbou pro použití v~kontextu SMT řešičů. Theory solver pro DPLL($T$) by měl efektivně podporovat dvě zásadní operace --- inkrementální přidávání literálů a~backtracking.

Na rozdíl od základního hladového přístupu, jak byl popsán v~\ref{smt}, v~DPLL($T$) dostává $Solver_T$ informaci o~rozhodnutých ohodnoceních průběžně. Aby byly dříve odhaleny slepé větve v~rozhodovacím stromu, $Solver_T$ je průběžně dotazován na splnitelnost dosud provedených rozhodnutí. Pro zrychlení tohoto procesu je tedy vhodné, aby byl schopen pro výpočet využít výsledků z~předchozích dokončených výpočtů. Algoritmy, které nedokáží takto zakomponovat mezivýpočet podproblému, budou ze své podstaty méně výkonné na postupné sekvenci splnitelných ohodnocení.

Po nalezení nesplnitelného ohodnocení pak neopakuje engine celý výpočet, ale vrací se pouze do nejbližšího stavu, ve kterém bylo ohodnocení ještě splnitelné. Od theory solveru očekáváme, že je schopen efektivně zapomínat přidané literály, vracet se do předchozích stavů a~pokračovat z~nich ve výpočtu. %% FIXME: Better wording?
S~ohledem na tyto požadavky vzniklo několik algoritmů pro řešení SMT s~ohledem na DL. V~této práci se budeme zabývat převážně postupem založeným na vyčerpávající propagaci teorie, který postulovali v~roce 2005 Nieuwenhuis a~Oliveras \cite{Nieuwenhuis05}.

\subsection{DPLL($T$) s~vyčerpávající propagací teorie}

Připomeňme si metodu \icode{SetTrue} uvedenou v~sekci \ref{smt}. Ta slouží k~přidání nového omezení do kontextu řešiče teorie. Řešič může volitelně jako návratovou hodnotu uvést nějakou množinu literálů, které detekoval jako důsledky vzniklé přidáním právě tohoto omezení. Může přitom hlásit např. jen \uv{zjevné} důsledky, popřípadě nemusí vracet vůbec žádné. Pokud jsme je ale schopni nacházet v~rozumém čase, nahlášené důsledky jsou užitečnou informací pro hlavní engine, poněvadž pro něj mohou výrazně zmenšit velikost rozhodovacího stromu.

Varianta DPLL($T$) s~vyčerpávající propagací teorie přidává pro \icode{SetTrue} silnou podmínku; řešič musí vrátit \emph{všechny} literály ze vstupní formule, které jsou důsledky stávajícího ohodnocení. Tento předpoklad značně zjednoduší DPLL($X$) engine a~umožní mu pracovat efektivněji. Stane se z~něj \emph{de~facto} běžný SAT řešič, který se liší pouze minimalistickým rozhranním se $Solver_T$. To má mimo jiné za důsledek, že jsme nyní schopni převézt libovolný SAT řešič na bázi DPLL do DPLL($X$) enginu.

Nevýhoda tohoto přístupu je jasná. Povinnost hledat všechny důsledky teorie může řádově zpomalit operaci \icode{SetTrue} pro řešič teorie. Nicméně se ukázalo, že alespoň v~případě diferenční logiky může tento přístup vést k~rychlé implementaci schopné předčit ostatní alternativy.

\subsection{Návrh řešiče pro diferenční logiku}

$Solver_T$ diferenční logiky využívá principy, se kterými jsme se podrobněji seznámili v~předchozích sekcích. Můžeme například předpokládat, že všechna omezení jsou tvaru $x-y \leq c$ (viz.~\ref{stp}). Využijeme též převodu omezení do tvaru omezujícího grafu, jak bylo uvedeno v~\ref{graf}.

\subsubsection*{Inicializace}
Během inicializace načte řešič vstupní formuli, uloží si všechna rozdílová omezení, která se v~ní vyskytují, a~předá ji DPLL($X$) jako čistě booleovskou formuli. Během tohoto procesu by měl detekovat vztahy mezi literály a~jejich negacemi a~explicitně je poznamenat. Pokud se například na vstupu vyskytnou nerovnice $x-y \leq 1$ a~$y-x \leq -2$, v~oboru celých čísel je jedna negací druhé. Řešič by tak měl abstrahovat tyto výskyty jako $p$ a~$\neg p$ pro booleovskou proměnnou $p$. Ukládá si přitom překladovou tabulku pamatující si pro každou nerovnici abstrahovaný literál, kterému odpovídá. Zároveň s~tím si udržuje pro každou proměnnou seznam všech nerovnic, ve kterých se tato proměnná vyskytuje (účel těchto seznamů je objasněn níže).

\subsubsection*{Ohodnocení literálu}
Jakmile je následně nastavena pravdivostní hodnota některého literálu, převede jej řešič do formy $x-y \leq c$ a~přidá odpovídající hranu do omezujícího grafu. Následně musí objevit všechny důsledky tohoto ohodnocení. Pro jejich nalezení je potřeba zkontrolovat všechny cesty $$x_i \xrightarrow{c_i*} x \xrightarrow{c} y \xrightarrow{c'_j*} y_j$$ a~zjistit, zda nějaké omezení neplyne z~$x_i - y_j \leq (c_i + c + c'_j)$. To budou právě nerovnice tvaru $x_i - y_j \leq c'$, kde $c' \geq c_i + c + c'_j$. 

Abychom prošli všechny tyto cesty, procházející nově přidanou hranou, musíme nejdřív nalézt seznam všech vrcholů $x_i$, ze kterých je dosažitelné $x$, a~seznam všech vrcholů $y_j$, které jsou dosažitelné z~$y$. Omezující graf je tedy reprezentován jako oboustranný seznam sousedů. Potom už můžeme spustit obyčejný algoritmus na hledání nejkratší cesty, kterým získáme všechna vhodná $x_i$ společně s~jejich $c_i$, resp. $y_j$ s~jejich $c'_j$. Autoři pro toto prohledávání doporučují následující postup.

Použijeme běžné prohledávání do hloubky. V~něm si označíme každý vrchol, který jsme navštívili poprvé, společně s~jeho vzdáleností. Navštívíme-li pak již objevený vrchol znovu, pokračujeme v~prohledávání pouze tehdy, když je jeho současná vzdálenost menší než naše uložená vzdálenost. Zároveň si všechny objevené $x_i$ a~$y_j$ ukládáme do dvou odlišných seznamů. Po skončení prohledávání spočteme pro oba seznamy počet všech nerovnic, ve kterých se proměnné v~seznamu vyskytují (tyto nerovnice si pro každou proměnnou pamatujeme z~inicializace).

Vyskytují-li se potom například $x_i$ celkově v~menším počtu proměnných, projdeme pro každé $x_i$ seznam všech nerovnic, ve kterých se vyskytuje, a~zkontrolujeme, zda se nejedná o~důsledek nalezené cesty z~$x_i$ do nějakého $y_j$.

\subsubsection*{Hledání příčin}

Jak víme z~\ref{smt}, řešiče teorie musí implementovat operaci \icode{Explain}, která pro nalezený důsledek vrátí množinu jeho příčin. Tato operace je důležitá pro budování implikačního grafu v~DPLL($X$) a~určení vhodné úrovně backtrackingu při nalezení sporu. Řešič DL popsaný v~\cite{Nieuwenhuis05} tuto operaci provádějí následovně.

Když je do omezujícího grafu přidána $n$-tá hrana, zapamatujeme si k~této hraně její odpovídající $n$. U~nalezených důsledků si pamatujeme $n$ hrany, jejíž přidáním důsledek vznikl. Když pak hledáme vysvětlení hrany $h$ tvaru $x-y \leq c$, spustíme prohledávání z~$x$ do $y$ stejným způsobem jako po přidání hrany. Prohledávání ale pouštíme jen do délky nejvýše $c$ a~jen přes hrany s~číslem vložení nejvýše $n$. Tato omezení zvyšují efektivitu (zmenšujeme prostor k~prohledání) a~zaručují, že nepoužíváme hrany přidané až po důsledku (což by působilo problémy v~implikačním grafu DPLL($X$)). Nalezená cesta tak má délku kratší nebo rovnou $c$ a~sestává se jen z~dříve přidaných hran, z~čehož jasně vidíme, že $h$ je důsledek hran na této cestě.

\subsection{Alternativní řešení}

TODO

