\chapter{Analýza}

\section{Fungování SMT řešičů}

\section{Rozbor STP}

Jedním ze základních podproblémů vyskytujícím se v~takřka všech plánovacích problémech je takzvaný Simple Temporal Problem (STP). STP poprvé postulovali \citet*{Dechter91} a~od té doby našel široká využití jak v~informatických oblastech, tak v~oborech od medicíny \citep{Anselma06} po vesmírný let \citep{Fukunaga97}.

Vstupem STP je množina rozdílových omezení, to jest nerovnic tvaru $$x - y \leq c,$$ kde $x$ a~$y$ jsou proměnné a~$c$ je konstanta. V~závislosti na tom, jakou verzi problému řešíme, pracujeme buď s~celočíselnými, nebo s~reálnými hodnotami. Výstupem tohoto problému je pak rozhodnutí, zda existuje ohodnocení proměnných tak, aby byla splněna všechna zadaná omezení. V~rozšíření problému pak můžeme požadovat na výstupu i~nějaké takovéto splnitelné ohodnocení, pokud existuje, případně nalezení pokud možno co nejmenší podmnožiny omezení, která zajišťují nesplnitelnost problému.

Na první pohled se může zdát pevně daný tvar nerovnic příliš omezující, uvědomme si však, že do této formy můžeme převést několik dalších druhů nerovnic. Nejsnáze zahrneme do problému omezení tvaru $x - y = c$; ty stačí jednoduše nahradit nerovnicemi $x - y \leq c$ a~$x - y \geq c$.

Problémy nedělají ani nerovnice typu $\pm x \leq c$. Pro účely takovýchto omezení si zavedeme novou globální proměnnou $zero$, s~jejíž pomocí převedeme předchozí do tvaru $x - zero \leq c$, respektive $zero - x \leq c$. Pokud pak hledáme splňující ohodnocení proměnných, najdeme takové, kde $zero$ je ohodnoceno nulou. Korektnost tohoto postupu zaručuje následující tvrzení.

\begin{lemma}
	Je-li $\sigma$ splňující ohodnocení nějakého STP a~$\varepsilon$ libovolná konstanta, pak ohodnocení $\pi$ definované pro všechny proměnné $x$ jako $\pi(x) = \sigma(x) + \varepsilon$ je také splňující ohodnocení tohoto STP.
\end{lemma}
\begin{proof}
	Plyne okamžitě z~tvaru rozdílových omezení.
\end{proof}

Můžeme do problému zahrnout taktéž omezení tvaru $x - y < c$. Pro celočíselné proměnné lze tuto nerovnici ekvivalentně zapsat jako $x - y \leq c-1$. V~reálné variantě pak nahradíme nerovnici výrazem $x - y \leq c - \delta$, přičemž nenastavujeme okamžitě konkrétní hodnotu $\delta$, ale udržujeme si ji pouze symbolicky a~určujeme její vhodné dosazení až při výpočtu splňujícího ohodnocení. Tento postup je detailněji popsán v~sekci \ref{int_v_real}. Uvědomme si, že pokud jsme schopni vyjádřit ostré nerovnosti, umíme vyjádřit i~negace neostrých nerovností a~naopak.

V~jazyce výrokové logiky pak teorii obsahující výše popsané nerovnice nazveme \emph{teorie diferenciální logiky} a~budeme ji značit $DL$. Celočíselnou variantu této teorie pak budeme značit jako $IDL$ a~reálnou variantu jako $RDL$. Nahradíme-li pak v~booleovské formuli některé termy těmito nerovnicemi, ověření splnitelnosti takto vzniklé formule je instancí SMT problému s~ohledem na $DL$.

\section{Převod na grafový problém}


\section{Volba algoritmu}

