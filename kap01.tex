\chapter{Analýza}

\section{Fungování SMT řešičů}

\section{Rozbor STP}

Jedním ze základních podproblémů vyskytujícím se v~takřka všech plánovacích problémech je takzvaný Simple Temporal Problem (STP). STP poprvé postulovali v~roce 1991 Dechter, Meiri a Pearl \cite{Dechter91} a~od té doby našel široká využití jak v~informatických oblastech, tak v~oborech od medicíny \cite{Anselma06} po vesmírný let \cite{Fukunaga97}.

Vstupem STP je množina rozdílových omezení, to jest nerovnic tvaru $$x - y \leq c,$$ kde $x$ a~$y$ jsou proměnné a~$c$ je konstanta. V~závislosti na tom, jakou verzi problému řešíme, přitom pracujeme buď s~celočíselnými, nebo s~reálnými hodnotami. Výstupem tohoto problému je pak rozhodnutí, zda existuje ohodnocení proměnných tak, aby byla splněna všechna zadaná omezení. V~rozšíření problému pak můžeme požadovat na výstupu i~nějaké takovéto splnitelné ohodnocení, pokud existuje, případně nalezení pokud možno co nejmenší podmnožiny omezení, která zajišťuje nesplnitelnost problému.

Na první pohled se může zdát pevně daný tvar nerovnic příliš omezující, uvědomme si však, že do této formy můžeme převést několik dalších druhů nerovnic. Nejsnáze zahrneme do problému omezení tvaru $x - y = c$; ty stačí jednoduše nahradit nerovnicemi $x - y \leq c$ a~$x - y \geq c$.

Problémy nedělají ani nerovnice typu $\pm x \leq c$. Pro účely takovýchto omezení si zavedeme novou globální proměnnou $zero$, s~jejíž pomocí převedeme předchozí do tvaru $x - zero \leq c$, respektive $zero - x \leq c$. Pokud pak hledáme splňující ohodnocení proměnných, najdeme takové, kde $zero$ je ohodnoceno nulou. Korektnost tohoto postupu zaručuje následující tvrzení.

\begin{lemma}
	Je-li $\sigma$ splňující ohodnocení nějakého STP a~$\varepsilon$ libovolná konstanta, pak ohodnocení $\pi$ definované pro všechny proměnné $x$ jako $\pi(x) = \sigma(x) + \varepsilon$ je také splňující ohodnocení tohoto STP.
\end{lemma}
\begin{proof}
	Plyne okamžitě z~tvaru rozdílových omezení.
\end{proof}

Můžeme do problému zahrnout taktéž omezení tvaru $x - y < c$. Pro celočíselné proměnné lze tuto nerovnici ekvivalentně zapsat jako $x - y \leq c-1$. V~reálné variantě pak nahradíme nerovnici výrazem $x - y \leq c - \delta$, přičemž nenastavujeme okamžitě konkrétní hodnotu $\delta$, ale udržujeme si ji pouze symbolicky a~určujeme její vhodné dosazení až při výpočtu splňujícího ohodnocení. Tento postup je detailněji popsán v~sekci \ref{int_v_real}. Uvědomme si, že pokud jsme schopni vyjádřit ostré nerovnosti, umíme vyjádřit i~negace neostrých nerovností a~naopak.

V~jazyce výrokové logiky pak teorii obsahující výše popsané nerovnice nazveme \emph{teorie diferenciální logiky} a~budeme ji značit \emph{DL}. Celočíselnou variantu této teorie pak budeme značit jako \emph{IDL} a~reálnou variantu jako \emph{RDL}. Nahradíme-li pak v~booleovské formuli některé termy těmito nerovnicemi, ověření splnitelnosti takto vzniklé formule je instancí SMT problému s~ohledem na DL.



\section{Převod na grafový problém}

Velkou rozšířenost STP můžeme mimo jiné přisoudit tomu, že jsme schopni ho efektivně řešit. Jelikož se problém skládá výlučně z~lineárních omezení, mohli bychom na první pohled využít metod lineárního programování, jako je například simplexový algoritmus. Tyto metody jsou schopny řešit i~mnohem komplexnější problémy, avšak s~jejich výpočetní silou se pojí znatelně vyšší časová náročnost. Algoritmy specializované na STP se proto už od svého počátku \cite[Kapitola 2]{Dechter91} obracejí jiným směrem, a~to k~formalizmu teorie grafů. Přestože v~průběhu let vzikly různé metody řešení tohoto problému, všechny fungují na základě převedení množiny omezení na takzvaný \emph{omezující graf}.

\begin{definice}[Omezující graf]
	Nechť $\Pi$ je množina rozdílových omezení. Omezujícím grafem této množiny rozumíme hranově ohodnocený orientovaný graf G takový, že vrcholy G tvoří proměnné vyskytující se v~$\Pi$ a~každému omezení $(x-y \leq c) \in \Pi$ odpovídá v~G hrana $\langle x,y\rangle$ s~ohodnocením $c$.
\end{definice}
\begin{pozn}
	Hranu $\langle x,y\rangle$ s~ohodnocením $c$ budeme značit $x \xrightarrow{c} y$. Orientovanou cestu z~$x$ do $y$ se součtem ohodnocení $k$ pak budeme značit $x \xrightarrow{k*} y$.
\end{pozn}

Pro úplnost dodejme, že dvojice proměnných se může vyskytovat v~libovolně mnoha omezeních. Omezující graf je tedy formálně orientovaným multigrafem. Vzhledem k~vzájemné bijekci mezi hranami grafu a~nerovnicemi problému budeme v~průběhu práce volně přecházet mezi oběma reprezentacemi.

Převod do formy grafu je pro řešení problému zásadní. Umožňuje nám totiž formulovat následující klíčové tvrzení.

\begin{tvrz}
	Nechť $\Pi$ je množina rozdílových omezení. Instance STP tvořená touto množinou je splnitelná právě tehdy, když omezující graf $\Pi$ neobsahuje záporné cykly.
\end{tvrz}
\begin{proof}
	Najdeme-li v~omezujícím grafu záporný cyklus obsahující vrchol $x$, sečtením všech nerovnic vyskytujících se v~tomto cyklu dostaneme $x-x \leq c < 0$, z~čehož je jasně problém nesplnitelný. Je-li na druhou stranu problém nesplnitelný, obsahuje $\Pi$ nějakou nerovnici $x - y \leq c$ takovou, že z~$\Pi$ vyplývá $y - x < -c$. Tato implikace znamená, že v~omezujícím grafu existuje cesta $y \xrightarrow{k*} x$ taková, že $k < -c$. Hrana $x \xrightarrow{c} y$ pak společně s~touto cestou tvoří záporný cyklus.
\end{proof}

Hledání splnitelnosti STP jsme tedy schopni převést na hledání záporného cyklu v~grafu. To je problém, který dokážeme efektivně řešit. Využít můžeme např. některý algoritmus na hledání nejkratší cesty, kupříkladu Floydův-Warshallův algoritmus operující v~čase $\Theta(\abs{V}^3)$ nebo Bellmanův-Fordův algoritmus, který má časovou složitost $\Theta(\abs{V}\cdot\abs{E})$.

Tyto algoritmy však trpí pro náš účel zásadním nedostatkem. Jejich použití znamená, že po každém přidání nové hrany do grafu musí znovu proběhnout celé prohledávání. Tento postup není vhodný pro použití v~SMT řešičích, ve kterých je kladen velký důraz na inkrementalitu. V~následující sekci tedy podrobně rozebereme několik postupů pro řešení problémů SMT s~ohledem na DL a~motivujeme výběr námi použitého algoritmu. 

\section{Volba algoritmu}

